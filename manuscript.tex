% !TeX root = RJwrapper.tex
\title{The Rockerverse: Packages and Applications for Containerization with R}
\author{by Daniel Nüst, Dirk Eddelbuettel, Dom Bennett, Robrecht Cannoodt, Dav Clark, Gergely Daroczi, Mark Edmondson, Colin Fay, Ellis Hughes, Sean Lopp, Ben Marwick, Heather Nolis, Jacqueline Nolis, Hong Ooi, Karthik Ram, Noam Ross, Lori Shepherd, Nitesh Turaga, Craig Willis, Nan Xiao, Charlotte Van Petegem}

\maketitle

\abstract{%
The Rocker~Project provides widely-used Docker images for R across
different application scenarios. This articles surveys downstream
projects building upon Rocker and presents the current state of R
packages for managing Docker images and controlling containers. These
use cases and the variety of applications demonstrate the power of
Rocker specifically and containerisation in general. We identified
common themes across this diversity: reproducible environments,
scalability and efficiency, and portability across clouds.
}

\hypertarget{introduction}{%
\section{Introduction}\label{introduction}}

\label{intro}

The R community keeps growing. This can be seen in the number of new
packages on CRAN, which keeps on growing exponentially
\citep{cran:2019}, but also in the numbers of conferences, open
educational resources, meetups, unconferences, and companies taking up,
as exemplified by the useR! conference
series\footnote{\href{https://www.r-project.org/conferences/}{https://www.r-project.org/conferences/}},
the global growth of the R and R-Ladies user
groups\footnote{\href{https://www.r-consortium.org/blog/2019/09/09/r-community-explorer-r-user-groups}{https://www.r-consortium.org/blog/2019/09/09/r-community-explorer-r-user-groups}, \href{https://www.r-consortium.org/blog/2019/08/12/r-community-explorer}{https://www.r-consortium.org/blog/2019/08/12/r-community-explorer}},
or the foundation and impact of the
R~Consortium\footnote{\href{https://www.r-consortium.org/news/announcements}{https://www.r-consortium.org/news/announcements}, \href{https://www.r-consortium.org/blog/2019/11/14/data-driven-tracking-and-discovery-of-r-consortium-activities}{https://www.r-consortium.org/blog/2019/11/14/data-driven-tracking-and-discovery-of-r-consortium-activities}}.
All this cements the role of R as the \emph{lingua~franca} of
statistics, data visualisation, and computational research. Coinciding
with the rise of R was the advent of
\href{https://en.wikipedia.org/wiki/Docker_(software)}{Docker} as a
general tool for distribution and deployment of server applications---in
fact, Docker can be called the \emph{lingua~franca} of describing
computing environments and packaging software. Combining both these
topics, the \emph{Rocker~Project}
(\url{https://www.rocker-project.org/}) provides images with R (see the
next Section for more details). The considerable uptake and continued
evolution of the Rocker~Project has lead to numerous projects extending
or building upon Rocker images, ranging from reproducible
research\textbackslash{}footnote\{``Reproducible'' in the sense of the
\emph{Claerbout/Donoho/Peng} terminology
\citep{barba_terminologies_2018}.\} to production deployments. This
article presents this \emph{Rockerverse} of projects across all
development stages: early demonstrations, working prototypes, and mature
products. We also introduce related activities connecting the R language
and environment with other containerisation solutions. The main
contributions is a coherent picture of the current lay of the land of
using containers in, with, and for R.

The article continues with a brief introduction of containerization
basics and the Rocker~Project, followed by use cases and applications,
starting with the R packages specificly for interacting with Docker,
second-level packages using containers indirectly or only for specific
features, up to complex use cases leveraging containers. We conclude
with a reflection on the landscape of packages and applications and
point out future directions of development.

\hypertarget{containerization-and-rocker}{%
\section{Containerization and
Rocker}\label{containerization-and-rocker}}

\label{containerisation} \label{rocker}

Docker, an application and service provide by the eponymous company, has
in just a few short years risen to prominence for development, testing,
deployment and distribution of computer software
\citep[cf.][]{datadog_8_2018,munoz_history_2019}. While there are
related approaches such as
LXC\footnote{\href{https://en.wikipedia.org/wiki/LXC}{https://en.wikipedia.org/wiki/LXC}}
or Singularity \citep{kurtzer_singularity_2017}, Docker has become
synomymous with ``containerization''---the method of taking software
artefacts and bundling them in such a way that use becomes standardized
and portable across operationg systems. In doing so, Docker had
recognised and validated the importance of one very important thread
that had been emerging, namely virtualization. By allowing (one or
possibly) multiple applications or services to run concurrently on one
host machine without any fear of interference between them, an important
scalability opportunity is being provided. But Docker improved this
compartmentalization by accessing the host system---generally
Linux---through a much thinner and smaller shim than a full operating
system emulation or virtualization. This containerization is also called
operating-system-level virtualization
\citep{wikipedia_contributors_os-level_2020}. Typically a container runs
one process, whereas virtualization may run whole operating systems at a
larger footprint. This makes for more efficient use of system resources
\citep{felter_updated_2015} and allowed another order of magnitude in
terms of scalability of deployment \citep[cf.][]{datadog_8_2018}. While
Docker makes use of Linux kernel features, its importance was large
enough so that some required aspects of running Docker have been added
to other operating systems to support Docker there more efficiently too
\citep{microsoft_linux_2019}. The success even lead to standardisation
and industry collaboration \citep{oci_open_2019}.

The key accomplishment of Docker as an ``application'' is to make a
``bundled'' aggregation of software (the so-called ``image'') available
to any system equipped to run Docker, without requiring much else from
the host besides the actual Docker application installation. This is a
rather attractive proposition and Docker's very easy to use user
interface has lead to widespread adoption and use of Docker in a variety
of domains, e.g., cloud computing infrastructure
\citep[e.g.,][]{Bernstein2014}, data science
\citep[e.g.,][]{boettiger_introduction_2015}, and edge computing
\citep[e.g.,][]{alam_orchestration_2018}. It provided to be a natural
match for ``cloud deployment'' which runs, or at least appears to run,
``seamlessly'' without much explicit reference to the underlying
machine, architecture or operating system: containers are portable and
can be deployed with very little in terms of dependencies on the host
system---only the container runtime is required. Images are normally
built from plain text documents called \texttt{Dockerfile}. A
\texttt{Dockerfile} has a specific set of instructions to create and
document a well-defined environment, i.e., install specific software and
expose specific ports.

For statistical computing and analysis centered around R, the
\textbf{Rocker~Project} has provided a variety of Docker containers
since its start in 2014 \citep{RJ-2017-065}. The Rocker~Project provides
several lines of containers spanning to from building blocks with
\texttt{R-release} or \texttt{R-devel}, via containers with
\href{https://rstudio.com/products/rstudio/}{RStudio~Server} and
\href{https://rstudio.com/products/shiny/shiny-server/}{Shiny~Server},
to domain-specific containers such as
\href{https://github.com/rocker-org/geospatial}{\texttt{rocker/geospatial}}
\citep{rocker_geospatial_2019}. These containers form \emph{image
stacks}, building on top of each other for better maintainability (i.e.,
smaller \texttt{Dockerfile}s), composability, and to reduce build time.
Also of note is a series of ``versioned'' containers which match the R
release they contain with the \emph{then-current} set of packages via
the MRAN Snapshot views of CRAN \citep{microsoft_cran_2019}. The
Rocker~Project's impact and importance was acknowledged by the Chan
Zuckerberg Initiative's \emph{Essential Open Source Software for
Science}, who provide findung for the projects's sustainable
maintenance, community growth, and targeting new hardware platforms
including GPUs \citep{chan_zuckerberg_initiative_maintaining_2019}.

Docker is not the only containerisation software. An alternative
stemming from the domain of high-perfomance computing is
\textbf{Singularity} \citep{kurtzer_singularity_2017}. Singularity can
run Docker images, and in the case of Rocker works out of the box if the
main process is R, e.g., in \texttt{rocker/r-base}, but does not succeed
in running images where there is an init script, e.g., in containers
that by default run RStudio~Server. In the latter case, a
\texttt{Singularity} file, a recipe akin to a \texttt{Dockerfile}, needs
to be used. To date, no comparable image stack to Rocker exists on
\href{https://singularity-hub.org/}{Singularity Hub}. A further tool for
running containers is
\href{https://github.com/containers/libpod}{\textbf{podman}}, which also
can build \texttt{Dockerfile}s and run Docker images. Proof of concepts
for using podman to build and run Rocker containers
exist\footnote{See \href{https://github.com/nuest/rodman}{https://github.com/nuest/rodman} and \href{https://github.com/rocker-org/rocker-versioned/issues/187}{https://github.com/rocker-org/rocker-versioned/issues/187}}.
Yet the prevelance of Docker, especially in the broader user community
beyond experts or niche systems, and the vast amount of blog posts and
courses for Docker, currently caps specific development efforts for both
Singularity and podman in the R community. This might quickly change
when usability and spread increase, or security features such as
rootless/unprivileged containers, which both these tools support out of
the box, become more sought after.

\hypertarget{use-cases-and-applications}{%
\section{Use cases and applications}\label{use-cases-and-applications}}

\label{applications}

\hypertarget{image-stacks-for-communities-of-practice}{%
\subsection{Image stacks for communities of
practice}\label{image-stacks-for-communities-of-practice}}

\textbf{Bioconductor} (\url{https://bioconductor.org/}) is an open
source, open development project for the analysis and comprehension of
genomic data \citep{gentleman_bioconductor_2004}. The project consists
of 1823 R software packages as of October 30th 2019, as well as packages
containing annotation or experiment data. \emph{Bioconductor} has a
semi-annual release cycle, each release is associated with a particular
version of R, and Docker images are provided for current and past
versions of \emph{Bioconductor} for convenience and reproducibility. All
images, included are described on the \emph{Bioconductor} web site (see
\url{https://bioconductor.org/help/docker/}), created with
\texttt{Dockerfile}s maintained on GitHub, and distributed through
Docker~Hub\footnote{See \href{https://github.com/Bioconductor/bioconductor_docker}{https://github.com/Bioconductor/bioconductor\_docker} and \href{https://hub.docker.com/u/bioconductor}{https://hub.docker.com/u/bioconductor} respectively.}.
\emph{Bioconductor}'s `base' Docker images are built on top of the
\texttt{rocker/rstudio} image. \emph{Bioconductor} installs packages
based on the R version in combination with the Bioconductor version, and
therefore uses Bioconductor version tagging \texttt{devel} and
\texttt{RELEASE\_X\_Y}, e.g., \texttt{RELEASE\_3\_10}. Past and current
combinations of R and \emph{Bioconductor} will therefore be accessible
via a specific image tags. The \emph{Bioconductor} \texttt{Dockerfile}
selects the desired version of R from Rocker, adds required system
dependencies, and uses the \CRANpkg{BiocManager} package for installing
appropriate versions of \emph{Bioconductor} packages. A strength of this
approach is that the responsibility for complex software configuration
(including customized development) is shifted from the user to the
experienced \emph{Bioconductor} core team. A recent audit of the
\emph{Bioconductor} image stack \texttt{Dockerfile} led to the
deprecation of several community-maintained images, because the numerous
specific images became too hard to understand, complex to maintain, and
cumbersome to extent. As part of the simplification, a recent innovation
is to produce a \texttt{bioconductor\_docker:devel} image to emulate the
\emph{Bioconductor} nightly Linux build machine as closely as possible.
This image contains the build system environment variables and the
\emph{system dependencies} needed to install and check almost all (1813
out of 1823) \emph{Bioconductor} software packages and saves users and
package developers from managing these themselves. Furthermore the image
is configured so that \texttt{.libPaths()} has
\texttt{/usr/local/lib/R/host-site-library} as the first location. Users
mounting a location on the host file system to this location can persist
installed packages across Docker sessions or image updates. Many R users
pursue flexible work flows tailored to particular analysis needs, rather
than standardized work flows. The new \texttt{bioconductor\_docker}
image is well-suited for this pattern, while
\texttt{bioconductor\_docker:devel} provides developers with a test
environment close to \emph{Bioconductor}'s build system.

\label{datascience} \textbf{Data Science} is a widely discussed topic
among all academic disciplines \citep[e.g.,][]{donoho_50_2017}. The
discussions shed a light on the tools and craftspersonship behind the
analysis of data with computational methods. The practice of Data
Science often involves a combination tools and software stacks and
requires a cross-cutting skillset. This complexity and an inherent
concern for openness and reproducibility in the Data Science community
lead to Docker being used widely. The remaineder of this section
presents exemplary Docker images and image stacks featuring R intended
for Data Science. The
\href{https://github.com/jupyter/docker-stacks/}{\emph{Jupyter Docker
Stacks}} project are a set of ready-to-run Docker images containing
Jupyter applications and interactive computing tools
\citep{project_jupyter_jupyter_2018}. The \texttt{jupyter/r-notebook}
image includes R and ``popular packages'', and naturally also the
IRKernel (\url{https://irkernel.github.io/}), an R kernel for Jupyter so
that Jupyter Notebooks can contain R code cells. R is also included in
the catchall \texttt{jupyter/datascience-notebook}
image\footnote{\href{https://jupyter-docker-stacks.readthedocs.io/en/latest/using/selecting.html}{https://jupyter-docker-stacks.readthedocs.io/en/latest/using/selecting.html}}.
For example, these images allow users to quickly start a Jupyter
Notebook server locally or build their own specialised images on top of
stable toolsets. R is installed using the Conda package manager, which
can manage environments for various programming languages, pinning both
the R version and the versions of R
packages\footnote{See \texttt{jupyter/datascience-notebook}'s \texttt{Dockerfile} at \href{https://github.com/jupyter/docker-stacks/blob/master/datascience-notebook/Dockerfile\#L47}{https://github.com/jupyter/docker-stacks/blob/master/datascience-notebook/Dockerfile\#L47}.}.
\emph{Kaggle} provides the
\href{https://hub.docker.com/r/kaggle/rstats}{\texttt{gcr.io/kaggle-images/rstats}}
image (previously \texttt{kaggle/rstats}) and
\href{https://github.com/Kaggle/docker-rstats}{corresponding
\texttt{Dockerfile}} for usage in their Machine Learning competitions
and easy access to the associated datasets. It includes machine learning
libraries such as Tensorflow and Keras (see also image
\texttt{rocker/ml} in Section~\nameref{rocker-gpu}), and also configures
the \CRANpkg{reticulate} package. The image uses a base image with
\emph{all packages from CRAN}, \texttt{gcr.io/kaggle-images/rcran},
which requires a Google Cloud Build as Docker~Hub would time
out\footnote{Originally, a stacked collection of over 20 images with automated builds on Docker~Hub was used, see \href{https://web.archive.org/web/20190606043353/http://blog.kaggle.com/2016/02/05/how-to-get-started-with-data-science-in-containers/}{https://web.archive.org/web/20190606043353/http://blog.kaggle.com/2016/02/05/how-to-get-started-with-data-science-in-containers/} and \href{https://hub.docker.com/r/kaggle/rcran/dockerfile}{https://hub.docker.com/r/kaggle/rcran/dockerfile}}.
The final extracted image size is over 25GB, which makes it debatable if
having everything available is actually convenient. As a further
example, \href{https://radiant-rstats.github.io/docs/}{\emph{Radiant}}
project provides several images, e.g.,
\href{https://hub.docker.com/r/vnijs/rsm-msba-spark}{\texttt{vnijs/rsm-msba-spark}},
for their browser-based business analytics interface based on
\CRANpkg{Shiny} (\texttt{Dockerfile}
\href{https://github.com/radiant-rstats/docker}{on GitHub}) and for use
in education as part of an MSc course. As data science often applies a
multitude of tools, this image favours inclusion over selection and
features Python, Postgres, JupyterLab and Visual Studio Code besides R
and RStudio, bringing the image size up to 9GB. \emph{Gigantum}
(\url{http://gigantum.com/}) is a platform for open and decentralized
data science with a focus on using automation and user-friendly tools
for easy sharing of reproducible computational workflows. Gigantum
builds on the \emph{Gigantum Client} (running either locally or on a
remote server) for development and execution of data-focused
\emph{Projects}, which can be stored and shared via the \emph{Gigantum
Hub} or via a zipfile export. The Client is a user friendly interface to
a backend using Docker containers to package, build, and run Gigantum
projects. It is configured to use a default set of Docker base images
(\url{https://github.com/gigantum/base-images}), and users are able to
define and configure their own custom images. The available images
include two with R based on Ubuntu Linux have the
\href{https://launchpad.net/~marutter/+archive/ubuntu/c2d4u3.5/}{\texttt{c2d4u}
CRAN PPA} pre-configured for installation of binary R
packages\footnote{\href{https://docs.gigantum.com/docs/using-r}{https://docs.gigantum.com/docs/using-r}}.
The R images vary in the included authoring environment, i.e., Jupyter
in \texttt{r-tidyverse} or both Jupyter \& RStudio in
\texttt{rstudio-server}. The independent image stack can be traced back
to the Gigantum environment and its features. The R images are based on
Gigantum's \texttt{python3-minimal} image, originally to keep the
existing front-end configuration, but also to provide a consistent
Python-to-R interoperability. The \texttt{Dockerfile}s also use build
args to specify bases, for example for different version of NVIDIA CUDA
for GPU
processing\footnote{See \href{https://github.com/gigantum/base-images/blob/master/_templates/python3-minimal-template/Dockerfile}{https://github.com/gigantum/base-images/blob/master/\_templates/python3-minimal-template/Dockerfile} for the \texttt{Dockerfile} of \texttt{python3-minimal}.}
so that appropriate GPU drivers can be enabled automatically when
supported. Furthermore, Gigantum's focus lies on environment management
via GUI and ensuring a smooth user interaction, e.g., with reliable and
easy conflict detection and resolution. For this reason, project
repositories store authoritative package information in a separate file
per package, allowing Git to directly detect conflicts and changes. A
Dockerfile is generated from this description that inherits from the
specified base image, and additional custom Docker instructions may be
appended by users, though Gigantum's default base images do not
currently include the \texttt{littler} tool, which is used by Rocker to
install packages within \texttt{Dockerfile}s. Because of these
specifics, instructions from \texttt{rocker/r-ubuntu} could \emph{not}
be readily re-used in this image stack (see
Section~\nameref{conclusions}). Both approaches enable \texttt{apt} as
an installation method, and this is exposed via the GUI-based
environment
management,\footnote{See \href{https://docs.gigantum.com/docs/environment-management}{https://docs.gigantum.com/docs/environment-management}}.
The image build and publication process is scripted with Python and JSON
template configuration files, unlike Rocker which relies on plain
\texttt{Dockerfile}s. A minor reason in the inception of the images were
also project constraints requiring a Rocker-incompatible licensing of
the \texttt{Dockerfile}s, i.e., the MIT License.

\hypertarget{interfaces-for-docker-in-r}{%
\subsection{Interfaces for Docker in
R}\label{interfaces-for-docker-in-r}}

\label{interfaces}

Interfacing with the Docker daemon is typically done through the
\href{https://docs.docker.com/engine/reference/commandline/cli/}{Docker
Command Line Interface} (Docker CLI). However, moving back and forth
between an R console and the command line can create friction in
workflows and reduce reproducibility. A number of first-order R packages
provide a interface to the Docker CLI, allowing to automate interaction
with the Docker CLI from an R console.

Each of these packages has particular advantages as they provide
function wrappers for interacting with the Docker CLI at different
stages of a container's life cycle. Examples of such interactions are
installing the Docker software, creating \texttt{Dockerfile}s
(\CRANpkg{dockerfiler}, \pkg{containerit}), building images and
launching a containers (\CRANpkg{stevedore}, \pkg{docker}) on a local
machine or on the cloud. As such, the choice of which package is most
useful depends on the use-case at hand, but also the users level of
expertise.

\begin{tabular}{l|l|l|l|l|l|l}
\hline
\rotatebox{-90}{Functionality} & \rotatebox{-90}{AzureContainers} & \rotatebox{-90}{babelwhale} & \rotatebox{-90}{dockermachine} & \rotatebox{-90}{dockyard} & \rotatebox{-90}{harbor} & \rotatebox{-90}{stevedore}\\
\hline
Generate a Dockerfile &  &  &  & \checkmark &  & \\
\hline
Build an image & \checkmark &  &  & \checkmark &  & \\
\hline
Execute a container locally or remotely & \checkmark & \checkmark & \checkmark & \checkmark & \checkmark & \checkmark\\
\hline
Deploy or manage an instances in the cloud & \checkmark &  & \checkmark &  & \checkmark & \checkmark\\
\hline
Interact with an instance (e.g., file transfer) &  & \checkmark & \checkmark &  &  & \checkmark\\
\hline
Manage storage of images &  &  &  &  & \checkmark & \checkmark\\
\hline
Supports Docker and Singularity &  & \checkmark &  &  &  & \\
\hline
Direct access to Docker API instead of using the CLI &  &  &  &  &  & \checkmark\\
\hline
Installing Docker software &  &  & \checkmark &  &  & \\
\hline
\end{tabular}

\textbf{\pkg{harbor}} (\url{https://github.com/wch/harbor} is not
actively maintained anymore, but should be honorably mentioned as the
first R package for managing Docker images and containers. It uses the
\CRANpkg{sys} package to run system commands against the Docker CLI,
both locally and through an SSH connection, and has convenience
functions, e.g., for listing and removing containers/images and for
accessing logs. The output of container executions are converted to
appropriate R types. The Docker CLI's basic functionality, while
evolving quickly and with small concern for avoiding breaking changes,
is unchanged for a long time so a core function such as
\texttt{harbor::docker\_run(image\ =\ "hello-world")} still works
depsite the stopped development.

\textbf{\CRANpkg{stevedore}}
(\url{https://cran.r-project.org/package=stevedore}) is currently the
most powerful Docker client in R. It interfaces with the Docker daemon
over the Docker HTTP
API\footnote{\href{https://docs.docker.com/engine/api/latest/}{https://docs.docker.com/engine/api/latest/}}
via a Unix socket on Linux or MacOS, over a named pipe on Windows, or
over an HTTP/TCP connection. The package is the only one not using
system calls to the \texttt{docker} CLI for managing images and
containers and easily exposes connections to remote Docker daemons,
which has to be configured on the Docker level otherwise. Using the API
gives access to more information and is system independent and likely
more reliable than parsing command line output. \CRANpkg{stevedore}'s
own interface is automatically generated based on the OpenAPI
specification of the Docker daemon, but still similar to the Docker CLI.
The interface is similar to R6 objects, in that a
\texttt{stevedore\_object} has a number of functions attached to it that
can be called, and multiple specific versions of the Docker API can be
supported thanks to the automatic
generation\footnote{See \href{https://github.com/richfitz/stevedore/blob/master/development.md}{https://github.com/richfitz/stevedore/blob/master/development.md}.}.

\textbf{\CRANpkg{AzureContainers}} is an interface to a number of
container-related services in Microsoft's
\href{https://azure.microsoft.com/}{Azure Cloud}
\citep{AzureContainers_2019}. While it is mainly intended for working
with Azure, as a convenience feature it includes lightweight,
cross-platform shells to Docker and Kubernetes (tools \texttt{kubectl}
and \texttt{helm}). These can be used to create and manage arbitrary
Docker images and containers, as well as Kubernetes clusters on any
platform or cloud service.

\textbf{\CRANpkg{babelwhale}} allows executing and interacting with
containers, which can use either Docker or Singularity as a backend
\citep{cannoodt_babelwhale_2019}. The package provides a unified
interface to interact with Docker and Singularity containers. Users can,
for example, execute a command inside a container, mount a volume or
copy a file.

\textbf{\pkg{dockyard}}
(\url{https://github.com/thebioengineer/dockyard}) has the goal of
lowering barrier to creating \texttt{Dockerfile}s, building Docker
images, and deploying Docker containers. The package follows the
increasingly used piping paradigm of the \texttt{tidyverse} style of
programming for chaining R functions representing the instructions in a
\texttt{Dockerfile}. An existing \texttt{Dockerfile} can be used as a
template. \pkg{dockyard} also includes wrappers for common steps, such
as installing an R package or copying files, and build-in functions for
building an image running a container, to make using Docker even more
approachable to R users with a single API.

\textbf{\pkg{dockermachine}}
(\url{https://github.com/cboettig/dockermachine}) is an R package to
provide a convenient interface to
\href{https://docs.docker.com/machine/overview/}{Docker~Machine} from R.
The CLI tool \texttt{docker-machine} allows users to create and manage
virtual host on local computers, local data centers, or at cloud
providers. A local Docker installation can be configured to transparntly
forward all commands issued on the local Docker CLI to a selected
(remote) virtual host. Docker~Machine was especially crucial for local
use in early days of Docker, when no native support was available for
Mac or Windows computers, but remains relevant for provisioning on
remote systems. The package has not received any updates for two years,
but is functional with a current version of \texttt{docker-machine}
(\texttt{0.16.2}). It potentially lowers the barriers for R users to run
containers on various hosts, if using the Docker~Machine CLI directly is
perceived as a barrier, or enables scripted workflows with remote
processing.

\hypertarget{capture-and-create-environments}{%
\subsection{Capture and create
environments}\label{capture-and-create-environments}}

\label{envs}

Several second order R packages attempt to make the process of creating
Docker images and using containers for specific tasks, such as running
tests or rendering reproducible reports, easier. While authoring and
managing an environment with Docker by hand is possible and feasible for
experts\footnote{See, e.g., this tutorial by RStudio on how to manage environments and package versions and to ensure deterministic image builds with Docker: \href{https://environments.rstudio.com/docker}{https://environments.rstudio.com/docker}.},
the following examples show the power of automation when environments
become too cumbersome. Especially \emph{version pinning}, with packages
\pkg{remotes} and \pkg{versions} for R or by using MRAN, and with system
package managers for different operating systems, can greatly increase
the reproducibility of built images and are common approaches.

\textbf{\pkg{dockerfiler}}
(\url{https://github.com/ColinFay/dockerfiler/}) is an R package
designed for building \texttt{Dockerfile}s straight from R. A scripted
creation of a \texttt{Dockerfile} enables iteration and automation, for
example for packaging applications for deployment (see
\nameref{deployment}). Being scriptable from R developers can leverage
the tools available in R to parse a \texttt{DESCRIPTION} file, to get
system requirements, to list dependencies, versions, etc.
\textbf{\pkg{containerit}}
(\url{https://github.com/o2r-project/containerit/}) attempts to take
this one step further and includes these tools to automatically create a
\texttt{Dockerfile} that can execute a given workflow
\citep{nust_containerit_2019}. \pkg{containerit} accepts R
\texttt{sessionInfo} objects as input and provides helper functions to
derive these from workflows, e.g., an R script or R Markdown document,
by analysing the session state at the end of the workflow. It relies on
the \pkg{sysreqs} (\url{https://github.com/r-hub/sysreqs/}) package and
it's mapping of package system dependencies to platform specific
installation package
names\footnote{See \href{https://sysreqs.r-hub.io/}{https://sysreqs.r-hub.io/}.}.
\pkg{containerit} uses \CRANpkg{stevedore} to streamline the user
interaction and improve the created \texttt{Dockerfile}s, e.g., by
running a container for the desired base image to extract the already
available R packages. \textbf{\pkg{dockr}}
(\url{https://github.com/smaakage85/dockr}) is a very similar package
attempting to mirror a given R session, including local non-CRAN
packages\citep{kjeldgaard_dockr_2019}. Users can manually add statements
for non-R dependencies to the \texttt{Dockerfile}.
\textbf{\CRANpkg{liftr}} aims to solve the problem of persistent
reproducible reporting in statistical computing based on the R Markdown
format \citep{xie2018} for dynamic documents
\citep[\href{https://nanx.me/liftr/}{https://nanx.me/liftr/}, ][]{liftr2019}.
The irreproducibility of authoring environments can become an issue for
collaborative documents and large-scale platforms for processing
documents. \CRANpkg{liftr} makes the document the main and sole workflow
control file and the only file that needs to be shared between
collaborators for consistent environments, e.g.~demonstrated in the
DockFlow project (\url{https://dockflow.org}). It introduces new fields
to the R Markdown document header, allowing users to manually declare
the dependencies, including versions, for rendering the document. The
package then generates a \texttt{Dockerfile} from this metadata and
provides a utitility function to render the document inside a Docker
container, i.e., \texttt{render\_docker("foo.Rmd")}. An RStudio addin
even allows compilation of documents with a single push of a button.

System dependencies are the domain of Docker, but for a full description
of the computing environment one must also manage the R version and the
the R packages. R~versions are available via the versioned Rocker image
stack \citep{RJ-2017-065}.
\href{https://github.com/ColinFay/ronline}{r-online} leverages these
images and provides an app for helping users to detect breaking changes
between different R versions, and for historic exploration of R. With a
standalone NodeJS app or
\href{https://srv.colinfay.me/r-online}{online}, the user can compare a
piece of code run in two separate versions of R. Internally, r-online
opens one or two Docker instances with the given version of R based on
Rocker images, executes a given piece of code, and returns the result to
the user. R~package management can be achieved with MRAN, or with
packages such as \CRANpkg{checkpoint} and \textbf{\CRANpkg{renv}}, which
can naturally be applied within images and containers. For example,
\CRANpkg{renv} (\url{https://rstudio.github.io/renv/}) helps users to
manage the state of the R library in a reproducible way, further
providing isolation and portability \citep{renv2019}. While
\CRANpkg{renv} does not cover system dependencies, the
\CRANpkg{renv}-based environment can be transferred into a container
either by restoring the environment based on the main configuration file
\texttt{renv.lock} or by storing the \CRANpkg{renv}-cache on the host
and not in the container \citep{ushey_using_2019}. With both the system
dependencies and R packages consciously managed in a Docker image, users
can start using containers as the \emph{only} environment for their
workflows, which allows them to work indepenently of physical
computers\footnote{Allowing them to be digital "nomads", cf. J.~Bryan's \href{https://github.com/jennybc/docker-why}{https://github.com/jennybc/docker-why}.}
and to assert a specific degree of confidence in the stability of a
developed software
\citep[cf. \texttt{README.Rmd} in][]{marwick_research_2017}.

\hypertarget{using-r-to-power-enterprise-software-in-production-environments}{%
\subsection{Using R to power enterprise software in production
environments}\label{using-r-to-power-enterprise-software-in-production-environments}}

\label{enterprise}

R has been historically viewed as a tool for analysis and scientific
research, but not for creating software that corporations can rely on
for production services. However, thanks to advancements in R running as
a web service, along with with the ability to deploy R in Docker
containers, modern enterprises are now capable of having real-time
machine learning powered by R. A number of packages and projects enabled
R to respond to client reqests over TCP/IP and local socket servers,
such as \CRANpkg{Rserve}, \CRANpkg{svSocket},
\href{http://www.rapache.net}{rApache} and more recently
\CRANpkg{plumber} (\url{https://www.rplumber.io/}) and \pkg{RestRserve}
(\url{http://restrserve.org}), which even processes incoming requests in
parallel with forked processes using \CRANpkg{Rserve}. The latter two
also provide documentation for deployment with Docker or ready to use
automated builds of
images\footnote{See \href{https://www.rplumber.io/docs/hosting.html\#docker}{https://www.rplumber.io/docs/hosting.html\#docker}, \href{https://hub.docker.com/r/trestletech/plumber/}{https://hub.docker.com/r/trestletech/plumber/} and \href{https://hub.docker.com/r/rexyai/restrserve/}{https://hub.docker.com/r/rexyai/restrserve/}.}.
These software allow other (remote) processes and programming languages
to interact with R and to expose R-based function in a service
architecture with HTTP APIs. APIs based on these package can be deployed
with scalability and high availability using containers. This pattern of
deploying code matches those used by software engineering services
created in more established languages in the enterprise domain, such as
Java or Python, and R can be used alongside those langages as a first
class member of a software engineering technical stack.

CARD.com implemented a web application for the optimization of the
acquisition flow and the real-time analysis of debit card transactions.
The software used \CRANpkg{Rserve} and rApache and was deployed in
Docker containers. The R session behind \CRANpkg{Rserve} acted as a
read-only in-memory database, which was extremely fast and scalabale,
for the many concurrent rApache processes responding to the live-scoring
requests of various divisions of the company. Similarly dockerized R
scripts were responsible for the ETL processes and even the
client-facing email, text message and push notification alerts sent in
real-time based on card transactions. The related Docker images were
made available at \url{https://github.com/cardcorp/card-rocker}. The
images extend \texttt{rocker/r-base} and additionally entailed an SSH
client and a workaround for being able to mount SSH keys from the host,
Pandoc, the Amazon Web Services (AWS) SDK, and Java, which is required
by the AWS SDK. The AWS SDK allowed to run R consumers reading from
real-time data processing streams of
\href{https://aws.amazon.com/kinesis/}{AWS Kinesis}
\footnote{See useR!2017 talk \href{https://static.sched.com/hosted\_files/user2017/2f/AWR Kinesis at useR 2017.pdf}{"Stream processing with R in AWS"}.}.
The applications were deployed on Amazon Elastic Container Service
(\href{https://aws.amazon.com/ecs/}{ECS}). The main learnings from using
R in Docker was the imporance of not only pinning the R package versions
via MRAN, but also moving away from Debian testing to a distribution
with long-term support. For the use case at hand, this switch served the
priority to have more control over upstream updates, and to minimize the
risk of breaking the automated builds of the Docker images and
production jobs.

The AI @ T-Mobile team created a set of neural network machine learning
natural language processing models to help customer care agents manage
text-based messages for customers \citep{t-mobile_enterprise_2018}. For
example, one model quickly identifies if a message is from a customer or
not
\citep[see \CRANpkg{Shiny}-based \href{https://secure.message.t-mobile.com/v1/shiny/is-customer/app/}{demo}, ][]{nolis_small_2019},
others tell which customers are likely to make a repeat purchase. If a
data scientist creates a machine learning model and exposes it through a
\CRANpkg{plumber} API, then someone else on the marketing team could
write software that sends different emails depending on that real-time
prediction. The models are convolutional neural networks that use the
\CRANpkg{keras} package and run in a Rocker Docker image. The
corresponding \texttt{Dockerfile}s are published
\href{https://github.com/tmobile/r-tensorflow-api}{on GitHub}. Since the
models power tools for agents and customers, they need to have extremely
high uptime and reliability. The AI @ T-Mobile team found that the
models performed well and today these models power real-time services
that are called over a million times a day.

\hypertarget{deployment-and-continuous-delivery}{%
\subsection{Deployment and continuous
delivery}\label{deployment-and-continuous-delivery}}

\label{deployment}

The cloud is the natural environment of containers, and subsequently
containers are the go-to mechanism to deploy R server applications. More
and more continuous integration (CI) and continuous delivery (CD)
services also use containers, opening up new options for use. The
controlled nature of containers, i.e., the possibility to abstract
internal software environment from a minimal dependency outside of the
container is crucial, for example to match test or build environments
with production environments or transfer runnable entities to
as-a-service infrastructures.

First, different packages use containers for the \textbf{deployment of R
and \CRANpkg{Shiny} apps}. \CRANpkg{Shiny} is a popular package for
creating interactive online dashboards with R and it enables users with
very diverse backgrounds to create stable and user friendly web
applications. For example, \emph{ShinyProxy}
(\url{https://www.shinyproxy.io/}) is an open source tool to deploy
Shiny apps in an enterprise context. They feature single sign-on, but
also in scientific use cases
\citep[e.g., ][]{savini_epiexplorer_2019,glouzon_structurexplor_2017}.
ShinyProxy uses Docker containers to isolate user sessions and to
achieve scalability for multi-user scenarious with multiple apps.
ShinyProxy itself is written in Java to accomodate corporate
requirements and may itself run in a container for stability and
availability. The tool is build on ContainerProxy
(\url{https://www.containerproxy.io/}), which provides similar features
for executign long-running R jobs or interactive R sessions. The started
containers can run on a regular Docker host but also in clusters.
Another example is the package \CRANpkg{golem}
(\url{https://github.com/ThinkR-open/golem}), which makes an heavy use
of \texttt{dockerfiler} when it comes to creating the
\texttt{Dockerfile} for building production-grade Shiny applications and
deploying them. \CRANpkg{googleComputeEngineR}
(\url{https://cloudyr.github.io/googleComputeEngineR/}) enables quick
deployments of key R services, such as RStudio and Shiny, onto cloud
virtual machines (VMs) with Google Cloud Compute Engine
\citep{googleComputeEngineR_2019}. The package utilises
\texttt{Dockerfile}s to move the labour of setting up those services
from the user to a premade Docker image, which is configured and run in
the cloud VM. For example, by specifying the template
\texttt{template="rstudio"} in functions
\texttt{gce\_vm\_template()/gce\_vm()} an up to date RStudio Service
image is launched for development work, while specifying
\texttt{template="rstudio-gpu"} will launch an RStudio Server image with
a GPU attached, etc.

Second, containers can be used to create \textbf{platform installation
packages} in a DevOps setting. The
\href{https://www.opencpu.org/}{OpenCPU} system provides an HTTP API for
data analysis based on R. \citet{ooms_opencpu_2017} describes how
various platform-specific installation files for OpenCPU are created
using Docker~Hub: the automated builds install the software stack from
source on different operating systems; afterwards a script file
downloads the images and extracts the OpenCPU binaries.

Third, containers can greatly facilitate the \textbf{deployment to
existing infrastructures}. \emph{Kubernetes}
(\url{https://kubernetes.io/}) is a container-orchestration system for
managing container-based application deployment and scaling. A
\emph{cluster} of containers, orchestrated as a single deployment, e.g.,
with Kubernetes, can mitigate limitations on request volumes or a
container occupied with computationally intensive task. A cluster
features load-balancing, autoscaling of containers across numerous
servers (in the cloud or on premise), and restarting failed ones. It may
be your organisation has a Kubernetes cluster already for other
applications, or you may use one of the different provides for managed
Kubernetes clusters. Docker containers are used within Kubernetes
clusters to hold native code, for which Kubernetes creates a framework
around network connections and scaling of resources up and down. R
applications, big parallel tasks, or scheduled batch jobs can be
deployed in a scalable way using Kubernetes, and the deployment can even
be triggered by changes to code repositories (i.e., CD), see blog post
``R on Kubernetes'' \citep{edmondson_r_2018}. The package
\pkg{googleKubernetesR}
(\url{https://github.com/RhysJackson/googleKubernetesR}) is a proof of
concept for wrapping the Google Kubernetes Engine API, Google's hosted
Kubernetes solution, in an easy to use R package. The package
\CRANpkg{analogsea} provides a solution to programmatically create and
destroy cloud VMs on the \href{https://www.digitalocean.com/}{Digital
Ocean} platform \citep{analogsea_2019}. It also includes R wrapper
functions to install Docker in such a VM, manage images, and control
containers straight from R functions. These functions are translated to
Docker CLI commands and transferred transparently to the respective
remove machine using SSH. \CRANpkg{AzureContainers} is an umbrella
package providing interfaces three commercial services of Microsoft's
Azure Cloud, namely
\href{https://azure.microsoft.com/en-us/services/container-instances/}{Container
Instances} for running individual containers,
\href{https://azure.microsoft.com/en-us/services/container-registry/}{Container
Registry} for private image distribution, and
\href{https://azure.microsoft.com/en-us/services/kubernetes-service/}{Kubernetes
Service} for orchestrated deployments. While a package like
\CRANpkg{plumber} provides the infrastructure for turning an R workflow
a service, for production purposes it is usually necessary to take into
account scalability, reliability and ease of management. AzureContainers
provides an R-based interface to these features and thereby simplifies
complex infrastructure management to a number of R function calls, given
an Azure account with sufficient
credit\footnote{See \emph{"Deploying a prediction service with Plumber"} vignette for details:  \href{https://cran.r-project.org/web/packages/AzureContainers/vignettes/vig01_plumber_deploy.html}{https://cran.r-project.org/web/packages/AzureContainers/vignettes/vig01\_plumber\_deploy.html}.}.
\href{https://www.heroku.com/}{Heroku} is a further cloud platform as a
service provider, who supports container-based applications.
\texttt{heroku-docker-r}
(\url{https://github.com/virtualstaticvoid/heroku-docker-r}) is an
independent project providing a template for deploying R applications
based on Heroku's image stack, including multiple examples for
interfacing R with other programming languages. Yet the approach
requires a manual management of the computing environment.

The prevelance of Docker in industry naturally leads to usage of
containers with R in such settings as well, as customers already manage
platforms in Docker containers. These products often entail a high
amount of open source software in combination with proprietary layers
adding the relevant commercializable features. One such example is
RStudio's data science platform
\href{https://rstudio.com/products/team/}{RStudio Team}. It allows teams
of data scientists and their respective IT/DevOps groups to develop and
deploy code in R and Python around the RStudio Open Source Server inside
of Docker images, without requiring users to learn new tools or directly
interact with containers. The best practices for
\href{https://support.rstudio.com/hc/en-us/articles/360021594513-Running-RStudio-with-Docker-containers}{running
RStudio with Docker containers} as well as
\href{https://github.com/rstudio/rstudio-docker-products}{Docker images}
for RStudio's commerical products are publicly available.

\hypertarget{common-or-public-work-environments}{%
\subsection{Common or public work
environments}\label{common-or-public-work-environments}}

\label{workenvs}

The fact that Docker images are portable and well defined make them
useful when more than one person needs access to the same computing
environment. This is even more useful when some of the users do not have
the expertise to create such an environment themselves, and when these
environments can be run in public or shared infrastructure.

The \href{https://mybinder.readthedocs.io/en/latest/}{\textbf{Binder}}
project, maintained by the team behind Jupyter, makes it possible for
users to \textbf{create and share computing environments} with others
\citep{jupyter_binder_2018}. A \emph{BinderHub} allows anyone with
access to a web browser and an internet connection to launch a temporary
instance of these custom environments and execute any workflows
contained within. From a reproducibility standpoint, Binder makes it
exceedingly easy to compile a paper, visualize data, and run small
examples from papers or tutorials without the need for any local
installation. To set up Binder for a project, a user typically starts at
an instance of a BinderHub and passes the location of a repository with
a workspace, e.g., a hosted Git repository, or a data repository like
Zenodo. Binder's core internal tool is \texttt{repo2docker}. It
deterministically builds a Docker image by parsing the contents of a
repository, e.g., project dependency configurations or simple
configuration
files\footnote{See supported file types at \href{https://repo2docker.readthedocs.io/en/latest/config\_files.html}{https://repo2docker.readthedocs.io/en/latest/config\_files.html}.}.
In the most powerful case, \texttt{repo2docker} builds a given
\texttt{Dockerfile}. While this approach works well for most run of the
mill Python projects, it is not so seamless for R projects. This is
partly because \texttt{repo2docker} does not support arbitrary base
images due to the complex auto-generation of the \texttt{Dockerfile}
instructions. Two approaches make using Binder easier. for R users.
First, \textbf{\pkg{holepunch}}
(\url{https://github.com/karthik/holepunch}) is an R package that was
designed to make sharing work environments accessible to novice R users
based on Binder. For any R projects that use the Tidyverse suite
\citep{wickham_welcome_2019}, the time and resources required to build
all dependencies from source can often time out before completion,
making it frustrating for the average R user. \pkg{holepunch} removes
some of these limitations by leveraging Rocker images that contain the
Tidyverse along special Jupyter dependencies, and only installs
additional packages from CRAN and Bioconductor that are not already part
of these images. It short cicuits the configuration file parsing in
\texttt{repo2docker} and starts with the Binder/Tidyverse base images,
which eliminates a large part of the build time and in most cases
results in a binder instance launching within a minute. \pkg{holepunch}
as a side effect also creates a \texttt{DESCRIPTION} file which then
turns any project into a research compendium
\citep{marwick_packaging_2018}. The \texttt{Dockerfile} included with
the project can also be used to launch a RStudio server locally, i.e.,
independent of Binder, which is especially useful when more or special
computational resources can be provided there. The local image usage
reduces the number of seperately managed environments and thereby
reduces work and increases portability and reproducibility. Second, the
\textbf{Whole~Tale} project (\url{https://wholetale.org}) combines the
strengths of the Rocker Project's curated Docker images with
\texttt{repo2docker}. Whole~Tale is a National Science Foundation (NSF)
funded project developing a scalable, open-source, multi-user platform
for reproducible research \citep{brinckman2019, chard2019a}. A central
goal of the platform is to enable researchers to easily create and
publish executable research objects\textbackslash{}footnote\{In
Whole\textasciitilde{}Tale a \emph{tale} is a research object that
contains metadata, data (by copy or reference), code, narrative,
documentation, provenance, and information about the computational
environment to support computational reproducibility\} associated with
published research \citep{chard2019b}. Using Whole~Tale, researchers can
create and publish Rocker-based reproducible research objects to a
growing number of repositories including DataONE member nodes, Zenodo
and soon Dataverse. Additionally, Whole~Tale supports automatic data
citation and is working on capabilities for image preservation and
provenance capture to improve the transparency of published
computational research artifacts \citep{mecum2018, mcphillips2019}. For
R users, Whole~Tale extends the Jupyter Project's \texttt{repo2docker}
component to simplify the customization of R-based environments for
researchers with limited experience with either Docker or Git. Multiple
options have been discussed to allow users to change the base image used
in \texttt{repo2docker} from the default Ubuntu LTS (long-term support)
required to support the Rocker Project images. Whole~Tale implemented a
custom
\texttt{RockerBuildPack}\footnote{See \href{https://github.com/whole-tale/repo2docker\_wholetale}{https://github.com/whole-tale/repo2docker\_wholetale}}
to support customization of the \texttt{rocker/geospatial} image through
\texttt{repo2docker}
composability\footnote{Composability refers to the ability to combine multiple package managers -- such as R, `pip`, and `conda`}.
This works in part because Rocker images are based on a Debian
distribution, so the instructions created by \texttt{repo2docker} for
Ubuntu work because of compatible shell and package manager.

In \textbf{high-performance computing}, one use for containers is to run
workflows on shared local hardware where teams manage their own
high-performance servers. This can follow one of several design
patterns: users may deploy containers to hardware as a work environment
for a specific project, conatiners may provide per-user persistent
environments, or a single container can act as a common multi-user
environment for a server. In all cases, though, the containerized
approach provides several advantages: First, users may use the same
image and thus work environment on desktop and laptop computers, as
well. The former models provide modularity, while the latter approach is
most similar to a simple shared server. Second, software updates can be
achieved by updating and redeploying the container, rather than tracking
local installs on each server. Third, the containerized environment can
be quickly deployed to other hardware, cloud or local, if more resources
are neccessary or in case of server destruction or failure. In any of
these cases, users need a method to interact with the containers, be it
and IDE, or command-like access and tools such as SSH, which is usually
not part of standard container recipes and must be added. The
Rocker~Project provides containers pre-installed with the RStudio~IDE.
In cases where users store nontrivial amounts of data for their
projects, data needs to persist beyond the life of the container. This
may be via in shared disks, attached network volumes, or in separate
storage where it is uploaded between sessions. In the case of shared
disks or network-attached volumes, care must be taken to persist user
permissions, and of course backups are still neccessary. When working
with multiple servers, an automation framework such as
\href{https://www.ansible.com}{Ansible} may be useful for managing
users, permisions, and disks along with containers.

\label{rocker-gpu} Using \textbf{GPUs} (graphical processing units) as a
specialised hardware from containerized common work environments is also
possible and useful \citep{haydel_enhancing_2015}. GPUs are increasingly
popular for compute-intensive machine learning (ML) tasks, e.g., deep
artificial neural networks \citep{schmidhuber_deep_2015}. Though in this
case, containers are not completely portable between hardware
environments, but the software stack for ML with GPUs is so complex to
set up that a ready-to-use container is helpful. Containers running GPU
software require drivers and libraries specific to GPU models and
versions, and containers require a specialized runtime to connect to the
underlying GPU hardware. For NVIDIA GPUs, the
\href{https://github.com/NVIDIA/nvidia-docker}{NVIDIA Container Toolkit}
includes a specialized runtime plugin for Docker and a set of base
images with appropriate drivers and libraries. The Rocker~Project
\href{https://github.com/rocker-org/ml}{has a repository} with (beta)
images based on these that include GPU-enabled versions of
machine-learning R packages, e.g., \texttt{rocker/ml} and
\texttt{rocker/tensorflow-gpu}.

Teaching is a further example where shared computing environments and
sandboxing can greatly improve the process. First, \textbf{prepared
environments for teaching} are especially helpful for (a) introductory
courses where the first step of installation and configuration can be
hurdles for students \citep{cetinkaya-rundel_infrastructure_2018}, and
(b) courses that require access to a relatively complex setup of
software tools, e.g., database systems.
\citet{cetinkaya-rundel_infrastructure_2018} describe how a Docker-based
deployment of RStudio (i) avoided problems with troubleshooting
individual students computers and greatly increase engagement through
very quickly showing tangible outcomes, e.g., a visualisation, and (ii)
reduced demand on teaching and IT staff. Each student get's acces to a
personal RStudio instance running in a container after authentication
with the university login, which gives the benefits of sandboxing and
possibility of limiting resources.
\citet{cetinkaya-rundel_infrastructure_2018} found that for the courses
at hand, actual usage of the UI is intermittent so a single cloud based
VM with 4~cores and 28GB~RAM sufficed for over 100 containers. An
example for mitigating \emph{complex setups} is teaching databases. R is
very useful tool for interfacing with databases, because almost every
open source and proprietary database system has an R package that allows
users to connect and interact with the it. This flexibility is even
broadened by \CRANpkg{DBI}, which allows to create a common API for
interfacing these databases, or the \CRANpkg{dbplyr} package, which runs
\CRANpkg{dplyr} code straight against the database as queries. But
learning and teaching these tools comes with the cost of deploying or
having access to an environment with the software and drivers installed.
For people teaching R, it can become a barrier if they need to install
local versions of database drivers, or to connect to remote instances
which might or might not be made available by IT services. Giving access
to a sandbox for the most common database environments is the idea
behind \href{https://github.com/ColinFay/r-db}{\texttt{r-db}}, a Docker
image that contains everything needed to connect to a database from R.
Notably, with \texttt{r-db}, the users don't have to install complex
drivers or to configure their machine in a specific way. The
\texttt{rocker/tidyverse} base image ensures that users can also readily
use packages for analysis, display, and reporting. Second, the idea of a
common environment and partitioning allow using \textbf{containers in
teaching for secure execution and automated testing} of submissions by
students. \href{https://dodona.ugent.be}{Dodona} is a web platform
developed at Ghent University and is used to teach students basic
programming skills, and it uses Docker containers to test submissions by
students. This means that both the code testing the students'
submissions and the submission itself are executed in a predictable
environment, avoiding compatibility issues between the wide variety of
configurations used by students. The containerization is also used to
shield the Dodona servers from bad or even malicious code: memory, time
and I/O~limits are used to make sure students can't overload the system.
The web application managing the containers communicates with them by
sending configuration information as a JSON document over standard
input. Every Dodona Docker image shares a \texttt{main.sh} file that
passes through this information to the actual testing framework, while
setting up some error handling. The testing process in the Docker
containers sends back the test results by writing a JSON document to its
standard output channel. In June 2019, R support was added to Dodona
using an image derived from the \texttt{rocker/r-base} image that sets
up the \texttt{runner} user and and \texttt{main.sh} file expected by
Dodona\footnote{\href{https://github.com/dodona-edu/docker-images/blob/master/dodona-r.dockerfile}{https://github.com/dodona-edu/docker-images/blob/master/dodona-r.dockerfile}}.
It also installs the packages required for the testing framework and the
exercises so that this doesn't have to happen every time a student's
submission is evaluated. The actual testing of R exercises is done using
a custom framework loosely based on \CRANpkg{testthat}. During the
development of the testing framework it was found that the
\CRANpkg{testthat} framework did not provide enough information to its
reporter system to send back all the fields required by Dodona to render
its feedback. Right now, multiple statistics courses are developing
exercises to automate the feedback for their lab classes.

\textbf{RCloud} (\url{https://rcloud.social}) is a cloud-based platform
for data analysis, visualisation and collaboration using R. It provides
a \texttt{rocker/drd} base image for easy evaluation of the
platform\footnote{\href{https://github.com/att/rcloud/tree/master/docker}{https://github.com/att/rcloud/tree/master/docker}}.

\hypertarget{processing}{%
\subsection{Processing}\label{processing}}

\label{processing}

The portability of containerized environments becomes particularly
useful for improving expensive processing of data or shipping complex
processing pipelines. First, it is possible to \textbf{offload complex
processing to a server} or clouds, and also parallel executing of
processes for speeding up or serving of many users.
\textbf{\CRANpkg{batchtools}} provides a parallel implementation of Map
for various schedulers \citep{Lang2017batchtools}. The package can
\href{https://mllg.github.io/batchtools/reference/makeClusterFunctionsDocker.html}{schedule
jobs with Docker Swarm} \textbf{\CRANpkg{googleComputeEngineR}} has the
function \texttt{gce\_vm\_cluster()} to create clusters of 2 or more
virtual machines, running multi-CPU architectures. Instead of running a
local R script with the local CPU and RAM restrictions, the same code
can be processed on all CPU threads of the cluster of machines in the
cloud, all running a Docker container with the same R environments.
\CRANpkg{googleComputeEngineR} integrates with the R parralisation
package \CRANpkg{future} to enable this with only a few lines or R
code\footnote{\href{https://cloudyr.github.io/googleComputeEngineR/articles/massive-parallel.html}{https://cloudyr.github.io/googleComputeEngineR/articles/massive-parallel.html}}.
\href{https://cloud.run}{Google Cloud Run} is a CaaS (Containers as a
Service) platform. Users can launch containers using any Docker image
without worrying about underlying infrastructure in a so called
serverless configuration. The service takes care of network ingress,
scaling machines up and down, authentication and authorisation---all
features which are non-trivial for a developer to create on their own.
This can be used to scale up R code to millions of instances if need be
with little or no changes to existing code, as demonstrated by the proof
of concept
\texttt{cloudRunR}\footnote{\href{https://github.com/MarkEdmondson1234/cloudRunR}{https://github.com/MarkEdmondson1234/cloudRunR}},
which uses Cloud Run to create a scalable R-based API using
\CRANpkg{plumber}. \href{https://cloud.google.com/cloud-build/}{Google
Cloud Build} and the Google Container Registry are a continuous
integration service respectively image registry that offload building of
images to the cloud, while serving the needs of commercial environments
such as private Docker images or image stacks. As Googl Cloud Build
itself can run any container, the package \pkg{googleCloudRunner}
demonstrates how R can be used as the control language for one time or
batch processing jobs and scheduling of
jobs\footnote{\href{https://code.markedmondson.me/googleCloudRunner/articles/cloudbuild.html}{https://code.markedmondson.me/googleCloudRunner/articles/cloudbuild.html}}.
\label{drake} \textbf{\CRANpkg{drake}} is a workflow manager for data
science projects \citep{landau_drake_2019}. It features implicit
parallel computing and automated detection of the parts of the work that
actually needs to be reexecuted. \CRANpkg{drake} has been demonstrated
to run inside containers for high
reproducibility\footnote{See for example \href{https://github.com/joelnitta/pleurosoriopsis}{https://github.com/joelnitta/pleurosoriopsis} or \href{https://gitlab.com/ecohealthalliance/drake-gitlab-docker-example}{https://gitlab.com/ecohealthalliance/drake-gitlab-docker-example}, the latter even running in a continuous integration platform (cf.~\nameref{development}.},
but also how to use \CRANpkg{future} package's function
\texttt{makeClusterPSOCK()} to send parts of the workflow to a Docker
image for
execution\footnote{\href{https://docs.ropensci.org/drake/index.html?q=docker\#with-docker}{https://docs.ropensci.org/drake/index.html?q=docker\#with-docker}}
\citep[see package's function documentation;~][]{future_2020}. In the
latter case, the container control code must be written by the user and
the \CRANpkg{future} package ensures the host and worker can connect for
communicating over socket connections. \textbf{RStudio Server Pro}
includes a functionality called
\href{https://solutions.rstudio.com/launcher/overview/}{Launcher} (since
version~1.2, released in 2019). It gives users the ability to spawn R
sessions and background/batch jobs in a scalable way on external
clusters, e.g.,
\href{https://support.rstudio.com/hc/en-us/articles/360019253393-Using-Docker-images-with-RStudio-Server-Pro-Launcher-and-Kubernetes}{Kubernetes
based on Docker images} or \href{https://slurm.schedmd.com/}{Slurm}
clusters, and optionally, with Singularity containers. A benefit of the
proprietary Launcher software is the ability for R and Python users to
leverage containerisation's advantages in RStudio without writing
specific deployment scripts or learning about Docker or managing
clusters at all.

\label{pipelines} Second, containers are perfectly suited for
\textbf{packaging and executing software pipelines} and required data.
Containers allow building complex processing pipelines that are
independent from the host programming language. Due to its original use
case (see~\nameref{introduction}), Docker has no standard mechanisms for
chaining containers together; it lacks definitions and protocols for
environment variables, volume mounts and/or ports that could enable
transfer of input (parameters and data) and output (results) to and from
containers. Some packages, e.g., \pkg{containerit}, do provide Docker
images that can be used similarly to CLIs, but their usage is
cumbersome\footnote{\href{https://o2r.info/containerit/articles/container.html}{https://o2r.info/containerit/articles/container.html}}.
\textbf{\pkg{outsider}} (\url{https://docs.ropensci.org/outsider/})
tackles the problem of integrating external programs into an R workflow
\citep{bennett_outsider_2020}. Installation and usage of external
programs can be difficult, convoluted and even impossible if the
platform is incompatible. Therefore \pkg{outsider} uses the
platform-independent Docker images to encapsulate processes in
\emph{outsider modules}. Each outsider module has a \texttt{Dockerfile}
and an R package with functions for interacting with the encapsulated
tool. Using only R functions, an end-user can install a module with the
\pkg{outsider} package and then call module code to integrate a tool
into their own R-based workflow seamlessly. The \pkg{outsider} package
and module manage the containers and handle the transmission of
arguments and the transfer of files to and from a container. These
functionalities also allow a user to launch module code on a remote
machine via SSH, expanding the potential computational scale. Outsider
modules can be hosted code-sharing services, e.g., on GitHub, and
\pkg{outsider} contains discovery functions for them.

\hypertarget{packaging-research-reproducibly}{%
\subsection{Packaging research
reproducibly}\label{packaging-research-reproducibly}}

\label{compendia}

Containers provide a high degree of isolation that is often desirable
when attempting to capture a specific computational environment so that
others can reproduce and extend a research result. Many computationally
intensive research projects depend on specific versions of original and
third-party software packages in diverse languages, joined together to
form a pipeline through which data flows. New releases of even just a
single piece of software in this pipeline can break the entire workflow,
making it difficult to find the error and difficult for others to reuse
existing pipelines. These breakages can make the original the results
irreproducible and not ¸ The chance of a substantial disruption like
this is high in a multi-year research project where key pieces of
third-party software may have several major updates over the duration of
the project. The classical ``paper'' article is insufficient to
adequately communicate the knowledge behind such research projects
\citep[cf.][]{donoho_invitation_2010,marwick_how_2015}.

\citet{gentleman_statistical_2007} coined the term \textbf{Research
Compendium} for a dynamic document together with supporting data and
code. They use the R package system \cite{core_writing_1999} as for the
functional prototype to structuring, validation, and distribution of
research compendia. This concept has been taken up and
extended\footnote{See full literature list at \href{https://research-compendium.science/}{https://research-compendium.science/}.},
not the least by applying containerisation and other methods for
managing computing environments---see Section~\nameref{envs}. Containers
give the researcher an isolated environment to assemble these research
pipelines with specific versions of software to minimize problems with
breaking changes and make workflows easier to share
\citep[cf.][]{boettiger_introduction_2015,marwick_packaging_2018}.
Research workflows in containers are safe from contamination from other
activities occuring on the researcher's computer, for example the
installation of newest version of packages for teaching demonstrations.
Given the users in this scenario, i.e., often academics with limited
formal software development training, templates and assistance with
containers around research compendia is essential. In many fields we see
that a typical unit of research for a container is a research report or
journal article, where the container holds the compendium, or
self-contained set of data (or connections to data elsewhere) and code
files needed to fully reproduce the article
\citep{marwick_packaging_2018}. The package \pkg{rrtools}
(\url{https://github.com/benmarwick/rrtools}) provides a template and
convienence functions to apply good practices for research compendia,
including a starter \texttt{Dockerfile}. Images of compendium containers
can be hosted on services such as Docker~Hub for convienient sharing
among collaborators and others. Similarly, packages such as
\pkg{containerit} and \pkg{dockerfiler} can be used to manage the
\texttt{Dockerfile} to be archived with a compendium on a data
repository (e.g.~\href{https://zenodo.org/}{Zenodo},
\href{https://dataverse.org/}{Dataverse},
\href{https://figshare.com/}{Figshare}, \href{https://osf.io/}{OSF}). A
typial compendium's \texttt{Dockerfile} will pull a rocker image fixed
to a specific version of R, and install R packages from the MRAN
repository to ensure the package versions are tied to a specific date,
rather than the most recent version. A more extreme case is the
\emph{dynverse} project (\url{https://dynverse.org/}), which packages
over 50 computational methods with different environments (R, Python,
C++, etc.) in Docker images, which can be executed from R.
\emph{dynverse} uses a CI platform (see~\nameref{ci}) to build
Rocker-derived images, test them, and publish them on Docker~Hub if the
tests succeed.

Future researchers can download the compendium from the repository and
run the included \texttt{Dockerfile} to build a new image that recreates
the computational environment used to produce the original research
results. If building the image fails, the human-readable instructions in
a \texttt{Dockerfile} are the starting point for rebuilding the
environment. When combinded with CI (see~\nameref{ci}), a research
compendium set-up can enable \emph{continuous analysis} with easier
verification of reproducibility and audits trails
\citep{beaulieu-jones_reproducibility_2017}.

Further safeguarding practices are currently under development or not
part of common practice yet, such as preservation of images
\citep{emsley_framework_2018} and storing them alongside
\texttt{Dockerfile}s \citep[cf.][]{nust_opening_2017}, or pinning of
system libraries beyond the tagged base images, which may be seen as
stable or dynamic depending on the applied time scale
\citep[see discussion on `debian:testing` base image in][]{RJ-2017-065}.
A recommendation of the recent National Academies' report on
\emph{Reproducibility and Replicability in Science} is that journals
\emph{``consider ways to ensure computational reproducibility for
publications that make claims based on computations''}
\citep{NASEM2019}. In fields such as Political Science and Economics,
journals are increasingly adopting policies requiring authors to publish
the code and data required to reproduce computational findings reported
in published manuscripts, subject to independent verification
\citep{Jacoby2017,Vilhuber2019,Alvarez2018,Christian2018,Eubank2016,King1995}.
Problems with the computational environment, installation and
availability of software dependencies are common. R is gaining
popularity in these communities, such as creating a research compendium.
In a sample of 105 replication packages published by the \emph{American
Journal of Political Science} (AJPS) over 65\% use R. The NSF-funded
Whole Tale project uses the Rocker Project community images with the
goal of improving the reproducibility of published research artifacts
and simplifying the publication and verification process for both
authors and reviewers by reducing errors and time spent specifying the
environment.

\hypertarget{development-debugging-and-testing}{%
\subsection{Development, debugging, and
testing}\label{development-debugging-and-testing}}

\label{development}

Containers can also serve as useful playgrounds to create environments
ad-hoc or to provide very specific environments that are not needed or
not easily available in day-to-day development for the purposes of
developing R packages. These environments may have specific versions of
R, of R extension packages, and of system libraries used by R extension
packages, and all of the above in a specific combination.

First, such containers can greatly facilitate \textbf{fixing bugs and
quick evaluation}, because developers and users can readily start and
later discard a container to investigate a bug report or try out a piece
of software \citep[cf.][]{ooms_opencpu_2017}. The container does not
affect their regular system. Using the Rocker images with RStudio, these
disposable environments lack no development comfort
(cf.~Section~\nameref{compendia}). \citet{ooms_opencpu_2017} describes
how \texttt{docker\ exec} can be used to get a root shell in a container
for cusotmization during software evaluation.
\citet{eddelbuettel_debugging_2019} describes an example how a Docker
container was used to debug an issue with a package only occuring with a
particular version of Fortran, and using tools which are not readily
available on all platforms (e.g., not on macOS).

Second, the strong integration of \textbf{system libraries in core
packages} in the \href{https://www.r-spatial.org/}{R-spatial community}
makes containers essential for stable and proactive development of
common classes for geospatial data modelling and analysis. For example,
GDAL \citep{gdal_2019} is a crucial library in the geospatial domain.
GDAL is a system dependency allowing R packages such as \CRANpkg{sf},
which provides the core data model for geospatial vector data, or
\CRANpkg{rgdal}, to accomodate users to be able to read and write
hundreds of different spatial raster and vector formats. \CRANpkg{sf}
and \CRANpkg{rgdal} have hundreds of indirect reverse imports and
dependencies and therefore the maintainers spend a lot of efforts not to
break these. Purpose built Docker are used to prepare for upcoming
releases of system libraries, individual bug reports, and for the lowest
supported versions of system
libraries\footnote{Cf. \href{https://github.com/r-spatial/sf/tree/master/inst/docker}{https://github.com/r-spatial/sf/tree/master/inst/docker}, \href{https://github.com/Nowosad/rspatial_proj6}{https://github.com/Nowosad/rspatial\_proj6}, and \href{https://github.com/r-spatial/sf/issues/1231}{https://github.com/r-spatial/sf/issues/1231}}.

Third, there are special purpose images for identifying problems beyond
the mere R code, such as \textbf{debugging R memory problems}. The
images significantly reduce the barrier to follow complex steps for
fixing memory allocation bugs
\citep[cf. Section~4.3 in][]{core_writing_1999}. These problems are hard
to debug and critical, both because when they occur they lead to fatal
crashing processes.
\href{https://github.com/rocker-org/r-devel-san}{\texttt{rocker/r-devel-san}}
and
\href{https://github.com/rocker-org/r-devel-san-clang}{\texttt{rocker/r-devel-ubsan-clang}}
are Docker images have a particularly configured version of R to trace
such problems with gcc and clang compilers, respectively
\citep[cf.~\CRANpkg{sanitizers} for examples,][]{eddelbuettel_sanitizers_2014}.
The image \href{https://github.com/wch/r-debug}{\texttt{wch/r-debug}} is
a purpose built Docker image with \emph{multiple} instrumented builds of
R, each with a different diagnostic utility activated.

Fourth, containers are useful for \textbf{testing} R code during
development. To submit a package to CRAN, an R package must work with
the development version of R, which must be compiled locally. That can
be a challenge for some users. The \textbf{R-hub} project provides
\emph{``a collection of services to help R package development''}, with
the package builder as the most prominent one\citep{r-hub_docs_2019}.
R-hub makes it easy to ensure that no errors occur, but to fix errors a
local setup is still often warranted, e.g., using the image
\texttt{rocker/r-devel}, and to test packages with native code, which
can make the process more complex \citep[cf.][]{eckert_building_2018}.
The R-hub Docker images can also be used to debug problems locally using
various combinations of Linux platforms, R versions, and
compilers\footnote{See \href{https://r-hub.github.io/rhub/articles/local-debugging.html}{https://r-hub.github.io/rhub/articles/local-debugging.html} and \href{https://blog.r-hub.io/2019/04/25/r-devel-linux-x86-64-debian-clang/}{https://blog.r-hub.io/2019/04/25/r-devel-linux-x86-64-debian-clang/}}.
The images go beyond the configurations, or \emph{flavours}, used by
CRAN for checking
packages\footnote{\href{https://cran.r-project.org/web/checks/check_flavors.html}{https://cran.r-project.org/web/checks/check\_flavors.html}},
e.g., with CentOS-based images, but lack a container for checking on
Windows or OS X. The images greatly support package developers to
provide support on operating systems they are not familiar. The package
\pkg{dockertest} (\url{https://github.com/traitecoevo/dockertest/}) is a
proof of concept for automatically generating \texttt{Dockerfile}s and
building images specifically to run
tests\footnote{\pkg{dockertest} is not actively maintained, but mentioned still because of its interesting approach.}.
These images are accompanied with a special launch script so the tested
source code is not stored in the image but the currently checked in
version from a local Git repository is cloned into the container at
runtime. This approach clearly seperates test environment, test code,
and current working copy of the code. \label{rselenium} Another use case
where a container helps to standardise tests across operating systems is
detailed the vignettes of the package \CRANpkg{RSelenium}
\citep{rselenium_2019}. The package recommends Docker for running the
\href{https://selenium.dev/}{Selenium} Server application needed to
execute test suites on browser-based user interfaces and webpages, but
requires users to manually manage the Docker containers.

\label{ci} Fifth, Docker images can be used \textbf{on continuous
integration (CI) platforms} to streamline the testing of packages.
\citet{ye_docker_2019} describes how they speed up the process of
testing by running tasks on \href{https://travis-ci.org/}{Travis~CI}
within a container using \texttt{docker\ exec}, e.g., the package check
or rendering of documentation. \citet{cardozo_faster_2018} saved time,
also on Travis~CI, by re-using the testing image as the base for an
image intended for publication on Docker~Hub.
\href{https://github.com/ColinFay/r-ci}{\texttt{r-ci}} is in turn used
with \href{https://docs.gitlab.com/ee/ci/}{GitLab~CI}, which itself is
built on top of Docker images: the user specifies a base Docker image,
and the whole tests are run inside this environment. The \texttt{r-ci}
image stack combines \texttt{rocker} versioning and a series of tools
specifically designed for testing in a fixed environment with a
customized list of preinstalled packages. Especially for long running
tests or complex system dependencies, these approaches to seperate
installation of build dependencies with code testing streamline the
development process. Not due to a concern about time, but to control the
environment used on a CI server, even this manuscript is rendered into a
PDF and deployed to a GitHub-hosted website with every change (see
\texttt{.travis.yml} and \texttt{Dockerfile} in the manuscript
repository). This gives on the one hand easy access after every update
of the R Markdown source code, and on the other hand a second controlled
environment making sure that the article renders successfully and
correctly.

\hypertarget{conclusions}{%
\section{Conclusions}\label{conclusions}}

\label{conclusions}

This article is a snapshot of the R-corner in a universe of applications
built with a many-faced piece of software, Docker. \texttt{Dockerfile}s
and Docker images are the go-to methods for collaboration between roles
in an organisation, such as development and IT operations teams, and
between parties in the communication of knowledge, such as research
workflows or education. Docker became synonymous with applying the
concept of containerisation to solve challenges of reproducible
environments, e.g., in research and in development \& production, and of
scalable deployments with the ability to move processing between
machines easily (e.g., locally, one cloud providers VM, another cloud
provider's Container-as-a-Service). Reproducible environments,
scalability and efficiency, and portability across
platforms/infrastructures are the common themes behind R packages, use
cases, and applications in this work.

The R packages and use cases presented show the growing number of users,
developers, and real-world applications in the community and the
resulting innovations. But the applications also point to the challenge
of keeping up with a continuously evolving landscape. The use cases
contributed by co-authors also have a degree of overlap, which can be
expected as a common language and understanding of good practices is
still taking shape. Also, the ease with which one can create a complex
software systems with Docker, such as an independent Docker image stack,
to serve one's specific needs leads to parallel developments. This
ease-of-DIY in combination with the difficulty to compose environments
based on \texttt{Dockerfile}s alone is a further reason for
\textbf{fragmentation}. Instructions can be outsourced into
distributable scripts and then copied into the image during build, but
that make \texttt{Dockerfile} hard to read and adds a layor of
complexity. Despite the different image stacks presented here, the
pervasiveness of Rocker can be traced back to its maintainers and the
user community valueing collaboration and shared starting points over
impulses to create individual solutions. Aside from that, fragmentation
may not be a bad sign, but instead a reflection of a growing market,
which is able to sustain multiple related efforts. With the maturing of
core building blocks, such as the Rocker suite of images, more systems
will be built successfully but will also be behind the curtains. Docker
alone, as a flexible core technology, is not a feasible level of
collaboration and abstraction. Instead, the use cases and applications
observed in this work provide a more useful division.

Nonetheless, at least on the level of R packages some
\textbf{consolidation} seems in order, e.g., to reduce the number of
packages creating \texttt{Dockerfile}s from R code or controlling the
Docker daemon with R code. It remains to be seen which approach to
control Docker, via the Docker API as \pkg{stevedore} or via system
calls as \pkg{dockyard}/\pkg{docker}/\pkg{dockr}, is more sustainable,
or if the question will be answered by the endurance of maintainers and
sufficient funding. Similarly, the capturing of environments and their
serialization in form of a \texttt{Dockerfile} currently happens at
different levels of abstraction and re-use of functionality seems
reasonable, e.g., \pkg{liftr} could generate the environment with
\pkg{containerit}, which in turn may use \pkg{dockerfiler} for low level
R objects representing a \texttt{Dockerfile} and its instructions. In
this consolidation, the Rocker~Project could play the role of
coordinating entity. Though for the moment, the sign of the times points
to more experimentation and feature growth, e.g., images for GPU-based
computing and artificial intelligence. Even with coding being more and
more accepted as a required, and achievable skill, an easier access, for
example by exposing containerisation benefits via simple user interfaces
in the users' IDE, could be an important next step since currently
containerisation happens more in the background for UI-based development
(e.g., a \texttt{rocker/rstudio} image in the cloud). Furthermore, a
maturing of the Rockerverse packages for managing containers may lead to
their adoption where currently manual coding is required, e.g.~in the
case of \CRANpkg{RSelentium} or \CRANpkg{drake} (see
Sections~\nameref{rselenium} and \nameref{drake} respectively). In
specific cases, e.g., for \CRANpkg{analogsea}, the interaction with the
Docker daemon may remain to specific to reuse first order packages to
control Docker.

New features, which make complex workflows accessible and reproducible,
and the variety in packages connected with containerisation, even when
they have overlapping features, are a signal and a support for a growing
user base. This growth is possibly the most important goal for the
foreseeable future in the \emph{Rockerverse}, and just like the Rocker
images have matured over years of use and millions of runs, the new
ideas and prototypes will have to proof themselves. It should be noted
that the dominant position of Docker is a blessing and a curse for these
goals. It could be wise to start experimenting with non-Docker
containerisation tools now, e.g., R packages interfacing with other
container engines such as
\href{https://github.com/containers/libpod}{podman/buildah} or
\href{https://coreos.com/rkt/}{rkt}, or an R package for creating
\texttt{Singularity} files. Such efforts can help to avoid lock-in and
to design sustainable workflows based on concepts of
\emph{containerisation}, not on their implementation in Docker. If
adoption of containerisation and R continue to grow, the missing pieces
for a success predominantly lie in (a) coordination and documentation of
activities to reduce repeated work in favour of open collaboration, (b)
the sharing of lessons learned from use cases to build common knowledge
and language, and (c) a sustainable continuation and funding for all of
development, community support, and education. A concrete effort to work
towards these pieces is to sustain the structure and captured status quo
from this work in form of a \emph{CRAN Task View on containerization}.

\hypertarget{author-contributions}{%
\section{Author contributions}\label{author-contributions}}

The ordering of authors following DN and DE is alphabetical. DN
conceived the article idea,
\href{https://github.com/nuest/rockerverse-paper/issues/3}{initialised the formation of the writing team},
wrote sections not mentioned below, and revised all sections. DB wrote
the section on \pkg{outsider}. GD contributed the CARD.com use case. DE
wrote the introduction and the section about Containerization and Rocker
and reviewed all sections. RC \& EH contributed to the section on
interfaces for Docker in R (\emph{dynverse} and \texttt{dynwrap}). DC
contributed content on Gigantum. ME contributed to the section on
processing and deployment to cloud services. CF wrote paragraphs about
\pkg{r-online}, \pkg{dockerfiler}, \pkg{r-ci} and \pkg{r-db}. SL
contributed content on RStudio's usage of Docker. BM wrote the section
on research compendia and made the project Binder-ready. HN \& JN
co-wrote the section on the T-Mobile use case. KR wrote the section
about \pkg{holepunch}. NR wrote paragraphs about shared work
environments and GPUs. LS \& NT wrote the section on Bioconductor. CW
wrote the sections on Whole Tale and contributed the publication
reproducibility audit use case. NX contributed content on \pkg{liftr}.
All authors approved the final version. This articles was
collaboratively written at
\href{https://github.com/nuest/rockerverse-paper/}{https://github.com/nuest/rockerverse-paper/}.
The
\href{https://github.com/nuest/rockerverse-paper/graphs/contributors}{contributors page},
\href{https://github.com/nuest/rockerverse-paper/commits/master}{commit history},
and
\href{https://github.com/nuest/rockerverse-paper/issues/}{discussion issues}
provide a detailed view on the respective contributions.

\hypertarget{acknowledgements}{%
\section{Acknowledgements}\label{acknowledgements}}

DN is supported by the project Opening Reproducible Research
(\href{https://www.uni-muenster.de/forschungaz/project/12343}{o2r})
funded by the German Research Foundation (DFG) under project number
\href{https://gepris.dfg.de/gepris/projekt/415851837}{PE~1632/17-1}. The
funders had no role in data collection and analysis, decision to
publish, or preparation of the manuscript. CW is supported by the Whole
Tale projects (\url{https://wholetale.org}) funded by the US National
Science Foundation (NSF) under award
\href{https://www.nsf.gov/awardsearch/showAward?AWD_ID=1541450}{OAC-1541450}.

\bibliography{RJreferences}


\address{%
Daniel Nüst\\
University of Münster\\
Institute for Geoinformatics\\ Heisenbergstr. 2\\ 48149 Münster, Germany\\ \orcid{0000-0002-0024-5046}\\
}
\href{mailto:daniel.nuest@uni-muenster.de}{\nolinkurl{daniel.nuest@uni-muenster.de}}

\address{%
Dirk Eddelbuettel\\
University of Illinois at Urbana-Champaign\\
Department of Statistics\\ Illini Hall, 725 S Wright St\\ Champaign, IL 61820, USA\\ \orcid{0000-0001-6419-907X}\\
}
\href{mailto:dirkd@eddelbuettel.com}{\nolinkurl{dirkd@eddelbuettel.com}}

\address{%
Dom Bennett\\
Gothenburg Global Biodiversity Centre, Sweden\\
Carl Skottsbergs gata 22B\\ 413 19 Göteborg, Sweden\\ \orcid{0000-0003-2722-1359}\\
}
\href{mailto:dominic.john.bennett@gmail.com}{\nolinkurl{dominic.john.bennett@gmail.com}}

\address{%
Robrecht Cannoodt\\
Ghent University\\
Data Mining and Modelling for Biomedicine group\\ VIB Center for Inflammation Research\\ Technologiepark 71\\ 9052 Ghent, Belgium\\ \orcid{0000-0003-3641-729X}\\
}
\href{mailto:robrecht@cannoodt.dev}{\nolinkurl{robrecht@cannoodt.dev}}

\address{%
Dav Clark\\
Gigantum, Inc.\\
1140 3rd Street NE\\ Washington, D.C. 20002, USA\\ \orcid{0000-0002-3982-4416}\\
}
\href{mailto:dav@gigantum.com}{\nolinkurl{dav@gigantum.com}}

\address{%
Gergely Daroczi\\
\\
\orcid{0000-0003-3149-8537}\\
}
\href{mailto:daroczig@rapporter.net}{\nolinkurl{daroczig@rapporter.net}}

\address{%
Mark Edmondson\\
IIH Nordic A/S, Google Developer Expert for GCP\\
\\
}
\href{mailto:mark@markedmondson.me}{\nolinkurl{mark@markedmondson.me}}

\address{%
Colin Fay\\
ThinkR\\
5O rue Arthur Rimbaud\\ 93300 Aubervilliers, France\\ \orcid{0000-0001-7343-1846}\\
}
\href{mailto:contact@colinfay.me}{\nolinkurl{contact@colinfay.me}}

\address{%
Ellis Hughes\\
Fred Hutchinson Cancer Research Center\\
Vaccine and Infectious Disease\\ 1100 Fairview Ave. N., P.O. Box 19024\\ Seattle, WA 98109-1024, USA\\
}
\href{mailto:ehhughes@fredhutch.org}{\nolinkurl{ehhughes@fredhutch.org}}

\address{%
Sean Lopp\\
RStudio, Inc\\
250 Northern Ave\\ Boston, MA 02210, USA\\
}
\href{mailto:sean@rstudio.com}{\nolinkurl{sean@rstudio.com}}

\address{%
Ben Marwick\\
University of Washington\\
Department of Anthropology\\ Denny Hall 230, Spokane Ln\\ Seattle, WA 98105, USA\\ \orcid{0000-0001-7879-4531}\\
}
\href{mailto:bmarwick@uw.edu}{\nolinkurl{bmarwick@uw.edu}}

\address{%
Heather Nolis\\
T-Mobile\\
12920 Se 38th St.\\ Bellevue, WA, 98006, USA\\
}
\href{mailto:heather.wensler1@t-mobile.com}{\nolinkurl{heather.wensler1@t-mobile.com}}

\address{%
Jacqueline Nolis\\
Nolis, LLC\\
Seattle, WA, USA\\ \orcid{0000-0001-9354-6501}\\
}
\href{mailto:jacqueline@nolisllc.com}{\nolinkurl{jacqueline@nolisllc.com}}

\address{%
Hong Ooi\\
Microsoft\\
Level 5, 4 Freshwater Place\\ Southbank, VIC 3006, Australia\\
}
\href{mailto:hongooi@microsoft.com}{\nolinkurl{hongooi@microsoft.com}}

\address{%
Karthik Ram\\
Berkeley Institute for Data Science\\
University of California\\ Berkeley, CA 94720, USA\\ \orcid{0000-0002-0233-1757}\\
}
\href{mailto:karthik.ram@berkeley.edu}{\nolinkurl{karthik.ram@berkeley.edu}}

\address{%
Noam Ross\\
EcoHealth Alliance\\
460 W 34th St., Ste. 1701\\ New York, NY 10001, USA\\ \orcid{0000-0002-0233-1757}\\
}
\href{mailto:ross@ecohealthalliance.org}{\nolinkurl{ross@ecohealthalliance.org}}

\address{%
Lori Shepherd\\
Roswell Park Comprehensive Cancer Center\\
Elm \& Carlton Streets\\ Buffalo, NY, 14263, USA\\ \orcid{0000-0002-5910-4010}\\
}
\href{mailto:lori.shepherd@roswellpark.org}{\nolinkurl{lori.shepherd@roswellpark.org}}

\address{%
Nitesh Turaga\\
Roswell Park Comprehensive Cancer Center\\
Elm \& Carlton Streets\\ Buffalo, NY, 14263, USA\\ \orcid{0000-0002-0224-9817}\\
}
\href{mailto:nitesh.turaga@roswellpark.org}{\nolinkurl{nitesh.turaga@roswellpark.org}}

\address{%
Craig Willis\\
University of Illinois at Urbana-Champaign\\
501 E. Daniel St.\\ Champaign, IL 61820, USA\\ \orcid{0000-0002-6148-7196}\\
}
\href{mailto:willis8@illinois.edu}{\nolinkurl{willis8@illinois.edu}}

\address{%
Nan Xiao\\
Seven Bridges Genomics\\
529 Main St, Suite 6610\\ Charlestown, MA 02129, USA\\ \orcid{0000-0002-0250-5673}\\
}
\href{mailto:me@nanx.me}{\nolinkurl{me@nanx.me}}

\address{%
Charlotte Van Petegem\\
Ghent University\\
Department WE02\\ Krijgslaan 281, S9\\ 9000 Gent, Belgium\\ \orcid{0000-0003-0779-4897}\\
}
\href{mailto:charlotte.vanpetegem@ugent.be}{\nolinkurl{charlotte.vanpetegem@ugent.be}}

