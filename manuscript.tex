% !TeX root = RJwrapper.tex
\title{The Rockerverse: Images, Packages, and Applications for Containerization
with R}
\author{by Daniel Nüst, Dom Bennett, Robrecht Cannoodt, Dav Clark, Gergely Daroczi, Dirk Eddelbuettel, Mark Edmondson, Colin Fay, Ellis Hughes, Sean Lopp, Ben Marwick, Heather Nolis, Jacqueline Nolis, Hong Ooi, Karthik Ram, Noam Ross, Lori Shepherd, Nitesh Turaga, Craig Willis, Nan Xiao, Charlotte Van Petegem}

\maketitle

\abstract{%
The Rocker~Project provides widely-used Docker images for R across
different application scenarios. This articles surveys downstream
projects building upon Rocker and presents the current state of R
packages for managing Docker images and controlling containers. We also
look beyond Rocker to other projects connecting containerisation with R,
namely alternative suites of images. These use cases and the variety of
applications demonstrate the power of Rocker specifically and
containerisation in general. We identified common themes across this
diversity: reproducible environments, scalability and efficiency, and
portability across clouds.
}

\hypertarget{introduction}{%
\section{Introduction}\label{introduction}}

\label{intro}

The R community keeps growing. This can be seen in the number of new
packages on CRAN, which keeps on growing exponentially
\citep{cran:2019}, but also in the numbers of conferences, open
educational resources, meetups, unconferences, and companies taking up,
as exemplified by the useR! conference
series\footnote{\href{https://www.r-project.org/conferences/}{https://www.r-project.org/conferences/}},
the global growth of the R and R-Ladies user
groups\footnote{\href{https://www.r-consortium.org/blog/2019/09/09/r-community-explorer-r-user-groups}{https://www.r-consortium.org/blog/2019/09/09/r-community-explorer-r-user-groups}, \href{https://www.r-consortium.org/blog/2019/08/12/r-community-explorer}{https://www.r-consortium.org/blog/2019/08/12/r-community-explorer}},
or the foundation and impact of the
R~Consortium\footnote{\href{https://www.r-consortium.org/news/announcements}{https://www.r-consortium.org/news/announcements}, \href{https://www.r-consortium.org/blog/2019/11/14/data-driven-tracking-and-discovery-of-r-consortium-activities}{https://www.r-consortium.org/blog/2019/11/14/data-driven-tracking-and-discovery-of-r-consortium-activities}}.
All this cements the role of R as the \emph{lingua~franca} of
statistics, data visualisation, and computational research. Coinciding
with the rise of R was the advent of
\href{https://en.wikipedia.org/wiki/Docker_(software)}{Docker} as a
general tool for distribution and deployment of server applications---in
fact, Docker can be called the \emph{lingua~franca} of describing
computing environments and packaging software. Combining both these
topics, the \emph{Rocker~Project}
(\url{https://www.rocker-project.org/}) provides images with R (see the
next Section for more details). The considerable uptake and continued
evolution of the Rocker~Project has lead to numerous projects extending
or building upon Rocker images, ranging from reproducible
research\textbackslash{}footnote\{``Reproducible'' in the sense of the
\emph{Claerbout/Donoho/Peng} terminology
\citep{barba_terminologies_2018}.\} to production deployments. This
article presents this \emph{Rockerverse} of projects across all
development stages: early demonstrations, working prototypes, and mature
products. We also introduce related activities connecting the R language
and environment with other containerisation solutions. The main
contributions is a coherent picture of the current lay of the land of
using containers in, with, and for R.

The article continues with a brief introduction of containerization
basics and the Rocker~Project, followed by descriptions of projects
developing images with R installations. Then, we present applications,
starting with the R packages specificly for interacting with Docker,
second-level packages using containers indirectly or only for specific
features, up to complex use cases leveraging containers. We conclude
with a reflection on the landscape of packages and applications and
point out future directions of development.

\hypertarget{containerization-and-rocker}{%
\section{Containerization and
Rocker}\label{containerization-and-rocker}}

\label{containerisation} \label{rocker}

Docker, an application and service provide by the eponymous company, has
in just a few short years risen to prominence for development, testing,
deployment and distribution of computer software
\citep[cf.][]{datadog_8_2018,munoz_history_2019}. While there are
related approaches such as
LXC\footnote{\href{https://en.wikipedia.org/wiki/LXC}{https://en.wikipedia.org/wiki/LXC}}
or Singularity \citep{kurtzer_singularity_2017}, Docker has become
synomymous with ``containerization''---the method of taking software
artefacts and bundling them in such a way that use becomes standardized
and portable across operationg systems. In doing so, Docker had
recognised and validated the importance of one very important thread
that had been emerging, namely virtualization. By allowing (one or
possibly) multiple applications or services to run concurrently on one
host machine without any fear of interference between them, an important
scalability opportunity is being provided. But Docker improved this
compartmentalization by accessing the host system---generally
Linux---through a much thinner and smaller shim than a full operating
system emulation or virtualization. This containerization is also called
operating-system-level virtualization
\citep{wikipedia_contributors_os-level_2020}. Typically a container runs
one process, whereas virtualization may run whole operating systems at a
larger footprint. This makes for more efficient use of system resources
\citep{felter_updated_2015} and allowed another order of magnitude in
terms of scalability of deployment \citep[cf.][]{datadog_8_2018}. While
Docker makes use of Linux kernel features, its importance was large
enough so that some required aspects of running Docker have been added
to other operating systems to support Docker there more efficiently too
\citep{microsoft_linux_2019}. The success even lead to standardisation
and industry collaboration \citep{oci_open_2019}.

The key accomplishment of Docker as an ``application'' is to make a
``bundled'' aggregation of software (the so-called ``image'') available
to any system equipped to run Docker, without requiring much else from
the host besides the actual Docker application installation. This is a
rather attractive proposition and Docker's very easy to use user
interface has lead to widespread adoption and use of Docker in a variety
of domains, e.g., cloud computing infrastructure
\citep[e.g.,][]{Bernstein2014}, data science
\citep[e.g.,][]{boettiger_introduction_2015}, and edge computing
\citep[e.g.,][]{alam_orchestration_2018}. It provided to be a natural
match for ``cloud deployment'' which runs, or at least appears to run,
``seamlessly'' without much explicit reference to the underlying
machine, architecture or operating system: containers are portable and
can be deployed with very little in terms of dependencies on the host
system---only the container runtime is required. Images are normally
built from plain text documents called \texttt{Dockerfile}. A
\texttt{Dockerfile} has a specific set of instructions to create and
document a well-defined environment, i.e., install specific software and
expose specific ports.

For statistical computing and analysis centered around R, the
Rocker~Project has provided a variety of Docker containers since its
start in 2014 \citep{RJ-2017-065}. The Rocker~Project provides several
lines of containers spanning to from building blocks with
\texttt{R-release} or \texttt{R-devel}, via containers with
\href{https://rstudio.com/products/rstudio/}{RStudio~Server} and
\href{https://rstudio.com/products/shiny/shiny-server/}{Shiny~Server},
to domain-specific containers such as
\href{https://github.com/rocker-org/geospatial}{\texttt{rocker/geospatial}}
\citep{rocker_geospatial_2019}. These containers form \emph{image
stacks}, building on top of each other for better maintainability (i.e.,
smaller \texttt{Dockerfile}s), composability, and to reduce build time.
Also of note is a series of ``versioned'' containers which match the R
release they contain with the \emph{then-current} set of packages via
the MRAN Snapshot views of CRAN \citep{microsoft_cran_2019}. The
Rocker~Project's impact and importance was acknowledged by the Chan
Zuckerberg Initiative's \emph{Essential Open Source Software for
Science}, who provide findung for the projects's sustainable
maintenance, community growth, and targeting new hardware platforms
including GPUs \citep{chan_zuckerberg_initiative_maintaining_2019}.

Docker is not the only containerisation software. An alternative
stemming from the domain of high-perfomance computing is
\textbf{Singularity} \citep{kurtzer_singularity_2017}. Singularity can
run Docker images, and in the case of Rocker works out of the box if the
main process is R, e.g., in \texttt{rocker/r-base}, but does not succeed
in running images where there is an init script, e.g., in containers
that by default run RStudio~Server. In the latter case, a
\texttt{Singularity} file, a recipe akin to a \texttt{Dockerfile}, needs
to be used. To date, no comparable image stack to Rocker exists on
\href{https://singularity-hub.org/}{Singularity Hub}. A further tool for
running containers is
\href{https://github.com/containers/libpod}{\textbf{podman}}, which also
can build \texttt{Dockerfile}s and run Docker images. Proof of concepts
for using podman to build and run Rocker containers
exist\footnote{See \href{https://github.com/nuest/rodman}{https://github.com/nuest/rodman} and \href{https://github.com/rocker-org/rocker-versioned/issues/187}{https://github.com/rocker-org/rocker-versioned/issues/187}}.
Yet the prevelance of Docker, especially in the broader user community
beyond experts or niche systems, and the vast amount of blog posts and
courses for Docker, currently caps specific development efforts for both
Singularity and podman in the R community. This might quickly change
when usability and spread increase, or security features such as
rootless/unprivileged containers, which both these tools support out of
the box, become more sought after.

\hypertarget{container-images}{%
\section{Container images}\label{container-images}}

\label{images}

\hypertarget{images-for-alternative-r-engines}{%
\subsection{Images for alternative R
engines}\label{images-for-alternative-r-engines}}

\label{alternatives}

As outlined above, R is a widely-used language with a large community.
The large number of extension packages provides access to an unrivaled
variety of established and upcoming features. Nevertheless, special use
cases and experimental projects in academia and industry exist to test
new approaches, or provide features different to what the R
implementation maintained by the
\href{https://www.r-project.org/contributors.html}{R Core Team} and
available via \href{https://cran.r-project.org/}{CRAN} provides. While
the images presented in this section are far from being as vetted,
stable, and widely used as any of the Rocker images, they demonstrate an
important advantage of containerisation technology, namely the abiliy to
transparently build portable stacks of open source software and make
them easily accessible to others. All these alternative implementations
of R are published under GPL licenses. Since all implementations of R
claim better performance as a core motivation, a comparision based on
Docker images, potentially leveraging the
\href{https://docs.docker.com/config/containers/resource_constraints/}{resource
restriction mechanisms} of Docker to level the playing field, seems
useful future work.

\textbf{Microsoft R Open} (MRO) is an R distribution formerly known as
Revolution R Open (RRO) before Revolution Analytics was acquired by
Microsoft. MRO is compatible with main R and its packages and it
``{[}..{]} includes additional capabilities for improved performance,
reproducibility, and platform support.'' \citep{microsoft_mro} Most
notably these capabilities are the MRAN repository, which is enabled by
default, and some bundled packages such as \CRANpkg{checkpoint} and
specific R packages maintained by
Microsoft\footnote{See "Bundled Packages" on \href{https://mran.microsoft.com/rro/installed}{https://mran.microsoft.com/rro/installed}.}.
Originally the optional integration with
\href{https://software.intel.com/en-us/mkl}{Intel® Math Kernel Library}
(MKL) for multi-threading in linear algebra operations
\citep{microsoft_multithread} was a distinguishing feature before it was
made available publicly, and can be used in base R, too
\citep{eddelbuettel_thinking_2018}.

MRO does not provide official Docker images, but a set of
community-maintained \texttt{Dockerfile}s and Docker images are provided
by the poject \texttt{mro-docker}
(\url{https://github.com/nuest/mro-docker}), e.g., the image
\href{https://hub.docker.com/repository/docker/nuest/mro/}{\texttt{nuest/mro}}.
The images are inspired by the Rocker images and can be used much in the
same fashion, effectively a drop-in replacement allowing users to
quickly evaluate if the benefits of MRO + Intel® MKL apply to their use
case. Version-tagged images are provided for the latest bugfix release
of recent R versions. Extended license information about MKL is printed
at every startup.

\textbf{Renjin} is an interpreter for the R language running on top of
the \href{https://en.wikipedia.org/wiki/Java_virtual_machine}{Java
Virtual Machine} (JVM) providing full two-way access between Java and R
code \citep{wikipedia_renjin_2018}. It was developed to combine the
benefits of R, such scripting and extension packages, with the JVM's
advantages in the areas of security, cross-platform availability and
integration into enterprise platforms. R extension packages need to be
specially compiled and are distributed via the Java package manager
\href{https://en.wikipedia.org/wiki/Apache_Maven}{Apache Maven},
cf.~\url{http://packages.renjin.org/packages} for available packages.
Packages are loaded on demand, i.e., at the first call to
\texttt{library()}. Not all R packages, especially one linking to binary
libraries, are available, e.g., \texttt{rgdal}
\footnote{\href{http://packages.renjin.org/package/org.renjin.cran/rgdal/1.4-4/build/1}{http://packages.renjin.org/package/org.renjin.cran/rgdal/1.4-4/build/1}}.
There are no offical Docker images for Renjin, but community-maintained
images for selected releases only are available under
\texttt{nuest/renjin} on Docker~Hub and GitHub at
\url{https://hub.docker.com/r/nuest/renjin} and
\url{https://github.com/nuest/renjin-docker} respectively. These images
expose the command line interface of Renjin in a similar fashion as
Rocker images and allow an easy evaluation of Renjin's suitability, but
are not intended for production use.

\textbf{pqR} (\url{http://www.pqr-project.org/}) is a ``a pretty quick
version of R''. \emph{pqR} attempts to improve R on some opinionated
issues in the R language and is the basis for experimental features,
e.g., automatic
diffentiation\footnote{\href{https://riotworkshop.github.io/abstracts/riot-2019-pqr.txt}{https://riotworkshop.github.io/abstracts/riot-2019-pqr.txt}}.
The source code development
\href{https://github.com/radfordneal/pqR/}{on GitHub} is a one man
project and it does not provide any Docker images. But especially
disruptive approaches may contribute to the development of the R
ecosystem, so the \texttt{nuest/pqr} image was independently created,
see Docker Hub at \url{https://hub.docker.com/r/nuest/pqr/} and GitHub
at \url{https://github.com/nuest/pqr-docker}.

\textbf{FastR} (\url{https://github.com/oracle/fastr}) is ``A
high-performance implementation of the R programming language, built on
GraalVM'' \citep{oracle_labs_oraclefastr_2020}. It is developed by
Oracle, connects R to the GraalVM ecosystem
\citep{wikipedia_graalvm_2019}, and also claims superior performance but
also targets full compatibility with base R
\citep{oracle_labs_oraclefastr_2020}. There are no offical Docker images
provided, but indepedently maintained experimental images and
\texttt{Dockerfile}s are provided by
\href{https://github.com/nuest/fastr-docker}{nuest/fastr-docker}, e.g.,
\href{https://cloud.docker.com/repository/docker/nuest/fastr/}{\texttt{nuest/fastr}
on Docker Hub}.

\hypertarget{bioconductor}{%
\subsection{Bioconductor}\label{bioconductor}}

\label{bioc}

\emph{Bioconductor} (\url{https://bioconductor.org/}) is an open source,
open development project for the analysis and comprehension of genomic
data \citep{gentleman_bioconductor_2004}. The project consists of 1823 R
software packages as of October 30th 2019, as well as packages
containing annotation or experiment data. \emph{Bioconductor} has a
semi-annual release cycle, each release is associated with a particular
version of R, and Docker images are provided for current and past
versions of \emph{Bioconductor} for convenience and reproducibility. All
images, included are described on the \emph{Bioconductor} web site (see
\url{https://bioconductor.org/help/docker/}), created with
\texttt{Dockerfile}s maintained on GitHub, and distributed through
Docker~Hub\footnote{See \href{https://github.com/Bioconductor/bioconductor_docker}{https://github.com/Bioconductor/bioconductor\_docker} and \href{https://hub.docker.com/u/bioconductor}{https://hub.docker.com/u/bioconductor} respectively.}.
\emph{Bioconductor}'s `base' Docker images are built on top of the
\texttt{rocker/rstudio} image. \emph{Bioconductor} installs packages
based on the R version in combination with the Bioconductor version, and
therefore uses Bioconductor version tagging \texttt{devel} and
\texttt{RELEASE\_X\_Y}, e.g., \texttt{RELEASE\_3\_10}. Past and current
combinations of R and \emph{Bioconductor} will therefore be accessible
via a specific image tags. The \emph{Bioconductor} \texttt{Dockerfile}
selects the desired version of R from Rocker, adds required system
dependencies, and uses the \CRANpkg{BiocManager} package for installing
appropriate versions of \emph{Bioconductor} packages. A strength of this
approach is that the responsibility for complex software configuration
(including customized development) is shifted from the user to the
experienced \emph{Bioconductor} core team. A recent audit of the
\emph{Bioconductor} image stack \texttt{Dockerfile} led to the
deprecation of several community-maintained images, because the numerous
specific images became too hard to understand, complex to maintain, and
cumbersome to extent. As part of the simplification, a recent innovation
is to produce a \texttt{bioconductor\_docker:devel} image to emulate the
\emph{Bioconductor} nightly Linux build machine as closely as possible.
This image contains the build system environment variables and the
\emph{system dependencies} needed to install and check almost all (1813
out of 1823) \emph{Bioconductor} software packages and saves users and
package developers from managing these themselves. Furthermore the image
is configured so that \texttt{.libPaths()} has
\texttt{/usr/local/lib/R/host-site-library} as the first location. Users
mounting a location on the host file system to this location can persist
installed packages across Docker sessions or image updates. Many R users
pursue flexible work flows tailored to particular analysis needs, rather
than standardized work flows. The new \texttt{bioconductor\_docker}
image is well-suited for this pattern, while
\texttt{bioconductor\_docker:devel} provides developers with a test
environment close to \emph{Bioconductor}'s build system.

\hypertarget{images-for-historic-r-versions}{%
\subsection{Images for (historic) R
versions}\label{images-for-historic-r-versions}}

\label{versions}

As with any other software, each new version of R comes with its share
of changes. Some are breaking changes, some are not, but the fact is
that running a piece of code in a given version might give different
results from another version can be an issue when it comes to
reproducibility of workflows and stability of applications. For example,
think about R random seed: starting with \texttt{R\ 3.6.0}, running
\texttt{set.seed(2811);\ sample(1:1000,\ 2)} will not give the same
result as if it was run inside an older version of R. That might seem
trivial, but all the code using the random number generator will not be
exactly reproducible after this breaking change. Therefore, controlling
the version of software is most crucial for reproducible research
\citep[e.g.][]{boettiger_introduction_2015}.

Containers are perfectly suited to capture a specific configuration of a
computing environment to prohibit such problems. The Rocker images
provides the \emph{versioned stack} for different the version of R since
the project inception, which use the \emph{then-current} stable Debian
base image \citep[cf.][]{RJ-2017-065}. Based on custom build phase
hooks\footnote{\href{https://docs.docker.com/docker-hub/builds/advanced/}{https://docs.docker.com/docker-hub/builds/advanced/}},
i.e., small shell scripts executed at different phases in the automated
build of Rocker images, there are also semantic version tags for the
most recent freezed (i.e., using MRAN) versions with only the major and
minor
version\footnote{See \href{https://github.com/rocker-org/rocker-versioned/blob/master/VERSIONS.md}{https://github.com/rocker-org/rocker-versioned/blob/master/VERSIONS.md}, FIXME: \href{https://github.com/rocker-org/rocker-versioned/issues/42}{https://github.com/rocker-org/rocker-versioned/issues/42}.}.
So, at the time of writing this article, \texttt{rocker/r-ver:3} and
\texttt{3.6} are aliases for \texttt{3.6.0}, because \texttt{3.6.1} is
the latest release. With the release of \texttt{3.6.2} and pinning of
the MRAN version in \texttt{3.6.1}, the tags \texttt{3} and \texttt{3.6}
will deliver the same image as \texttt{3.6.1}. These tags allow users to
update the base image to retrieve bugfixes while reducing the risk of
also introducing breaking changes.

\href{https://github.com/ColinFay/ronline}{r-online} is an app for
helping users to detect breaking changes between different R versions,
and for historic exploration of R. With a standalone NodeJS app or
\href{https://srv.colinfay.me/r-online}{online}, the user can compare a
piece of code run in two separate versions of R. Internally, r-online
opens one or two Docker instances with the given version of R based on
Rocker images, executes a given piece of code, and returns the result to
the user.

A container also provides an isolated sandbox environment suitable for
testing and evaluation, without interfering with the ``main'' working
environment. This enables cross version testing and bugfixing. To take
this even further, the Rocker contributors are discussing the provision
of R versions reaching back futher than the project's own inception,
reaching back as far as R \texttt{2.x} and \texttt{1.x}
\footnote{\href{https://github.com/rocker-org/rocker-versioned/issues/138}{https://github.com/rocker-org/rocker-versioned/issues/138}}.
The main challenges are finding a suitable base image with a matching OS
version, or making the adjustments to compile R on more recent OS
releases.

\hypertarget{windows-images}{%
\subsection{Windows Images}\label{windows-images}}

\label{windows}

Docker containers on the Windows operating system were originally quite
cumbersome to use. They required an extra tool,
\href{https://docs.docker.com/toolbox/}{Docker Toolbox}, which by now
only exists as a legacy solution for older Windows systems. Docker
toolbox leverages \texttt{docker-machine} (see also
Section~\nameref{interfaces} for an R package interfacing with
\texttt{docker-machine}) to handle the process of creating a local
virtual host which could host Docker Engine, while exposing the regular
Docker CLI. The Docker CLI commands are forwarded to the virtual host
transparently for the user.

For current Windows Server (2016 and later) and Windows Desktop
(\href{https://www.docker.com/products/docker-desktop}{Docker Desktop}
requires Windows 10) versions, Docker is supported
natively\footnote{\href{https://www.docker.com/products/windows-containers}{https://www.docker.com/products/windows-containers}}
and different base images are offered by
Microsoft\footnote{\href{https://docs.microsoft.com/en-us/virtualization/windowscontainers/about/index}{https://docs.microsoft.com/en-us/virtualization/windowscontainers/about/index}}.
The Docker CLI can be used in just the same way as on other operating
systems but not every base image is supported on every Windows
host\footnote{\href{https://docs.microsoft.com/en-us/virtualization/windowscontainers/deploy-containers/version-compatibility}{https://docs.microsoft.com/en-us/virtualization/windowscontainers/deploy-containers/version-compatibility}}.
On Docker Desktop for Windows, the user can run both Linux-based and
Windows-based containers, but only one of the Docker daemons can be used
at a
time\footnote{See "Switch between Windows and Linux containers" on \href{https://docs.docker.com/docker-for-windows/}{https://docs.docker.com/docker-for-windows/}.}.

\textbf{rocker-win} (\url{https://github.com/nuest/rocker-win}) is a
proof of concept for running R in Windows-based containers
\citep{nust_rocker-win_2019}. It provides selected R versions using
three different base images using \texttt{microsoft/windowsservercore}
and \texttt{mcr.microsoft.com/windows/servercore}. These base images
match different Windows versions: Windows Server 2019, Windows Server
2016, and Windows Server, version 1803. The images are built using two
CI services,
\href{https://docs.travis-ci.com/user/reference/windows/}{Travis CI} and
\href{https://www.appveyor.com/docs/windows-images-software/}{Appveyor},
which provide the different Windows versions and therefore support
different base images, and published automatically
\href{https://hub.docker.com/r/nuest/rocker-win}{on Docker~Hub}. The
images are built from manually maintained seperate \texttt{Dockerfile}s,
are tagged with both Windows-variant and R version, e.g.,
\texttt{nuest/rocker-win:ltsc2019-3.6.2} or
\texttt{nuest/rocker-win:1803-latest}, and run the \texttt{R.exe}
process by default. All these images can be run on Docker for Desktop on
Windows 10. New R versions are only added on demand. The explorative
rocker-win project demonstrates the variety of Windows on Docker, which
is potentially confusing, e.g., because of the different base images.
The proof of concept provides a starting point for any
Windows-constrained user to leverage containerisation, including
examples for apps using \CRANpkg{plumber}, \CRANpkg{Shiny}, or R
packages with system dependencies such as \CRANpkg{sf}. The latter is
quite comfortable and fast actually, compared to the default
installation from source on Linux, because CRAN ships pre-compiled
binaries for Windows. However, compared to the various Linux images, the
footprint of Windows images is quite large.

For Windows, just as originally for Linux, the cloud use cases drive the
development and Docker and Microsoft collaborate
closely\footnote{\href{https://www.docker.com/partners/microsoft}{https://www.docker.com/partners/microsoft}}.
In most cases, individual developers or researchers worrying about
reproducibility, will prefer the more widely and host independent
Linux-based containers. Naturally, a Windows license is required for the
host machine, and the licensing impacts the potential to build images in
clouds or redistribute exported images as archive files. But the
developments in the \texttt{rocker-win} prototype show that containers
can also be used for R workflows depending on Windows-only tools, for
leveraging an existing Windows Server-based infrastructure where
policies can otherwise not be met, or for streamlining interactions with
system operating staff that runs Windows-based containers themselves.

\hypertarget{non-rocker-linux-images}{%
\subsection{Non-Rocker Linux images}\label{non-rocker-linux-images}}

\label{nondebian}

\label{rhub} The \textbf{R-hub} project provides \emph{``a collection of
services to help R package development''}, with the package builder as
the most prominent one\citep{r-hub_docs_2019}. The builder allows R
package developers to check their R package on different platforms and R
versions using a web form or the package \CRANpkg{rhub}
\citep{csardi_rhub_2019}. The builder uses Docker containers to conduct
these checks, taking advantages of their well-defined environments and
sandboxing. The \texttt{Dockerfile}s and images are published on
Docker~Hub\footnote{\href{https://hub.docker.com/u/rhub}{https://hub.docker.com/u/rhub}}
and
GitHub\footnote{\href{https://github.com/r-hub/rhub-linux-builders}{https://github.com/r-hub/rhub-linux-builders}}
respectively. The images comprise release and development builds of R
running in the base images of the
\href{https://www.debian.org/}{Debian},
\href{https://ubuntu.com/}{Ubuntu},
\href{https://getfedora.org/}{Fedora}, and
\href{https://centos.org/}{CentOS} linux distributions (Arch Linux is
under
development\footnote{\href{https://github.com/r-hub/rhub-linux-builders/pull/41}{https://github.com/r-hub/rhub-linux-builders/pull/41}},
see \texttt{rhub::local\_check\_linux\_images()} for a full list), and
are intended for debugging (see Section~\nameref{development}) not for
reproducible workflows. \emph{The rationale} of not re-using the Rocker
image stack is the high number of operatings systems and configurations,
which can be more uniformly maintained in a seperate suite of images.
Also, a common structure across the respective image stacks of the
operating systems, e.g., for \texttt{debian}, adding GCC in
\texttt{debian-gcc} and then R's development version in
\texttt{debian-gcc-devel}. The platform also has a specific user
configuration differing from the setup covering typical applications of
Rocker images, and the images include considerably more software, such
as a fairly complete LaTeX installation, to provide an experience closer
to CRAN.

\href{https://rstudio.com/}{\textbf{RStudio}} also maintains a set of
\href{https://github.com/rstudio/r-docker}{Docker images} for multiple
operating systems including SUSE, Ubuntu, CentOS, and Debian. The image
stack includes an \href{https://github.com/rstudio/r-builds}{opinionated
R installation} and ensures the R installation and profile is consistent
across the different Linux distributions. The base image also installs R
in a versioned directory and uses a minimal set of build and runtime
dependencies. Furthermore these images are responsible for creating
pre-compiled binary R packages for Linux, which supplement the binaries
built by CRAN. These pre-compiled packages dramatically decrease the
installation time of R packages on Linux and subsequently can decrease
the build time of Docker images \citep{lopp_package_2019}.

There are also independent projects installing R or RStudio on
\href{https://www.alpinelinux.org/}{\textbf{Alpine Linux}}, but not
beyond the proof of concept
stage\footnote{\href{https://gitlab.com/artemklevtsov/r-alpine}{https://gitlab.com/artemklevtsov/r-alpine} or \href{https://www.github.com/velaco/alpine-r}{https://www.github.com/velaco/alpine-r} provide different use case images but are not recently maintained (development of the latter documented in \citet{ratesic_building_2018}); \href{https://github.com/cmplopes/alpine-r}{https://github.com/cmplopes/alpine-r} has only base R images, is relatively up-to-date but sparsely documented; \href{https://github.com/CenterForStatistics-UGent/mountainr}{https://github.com/CenterForStatistics-UGent/mountainr} is a ghost project with only a single short `Dockerfile`, yet it shows the easy installation of the latest R from Alpine sources}.
In general, Alpine-based Docker images are often chosen for minimalistic
and thereby small and secure containers, e.g., in hardware with limited
storage. R-hub's proof of concept
\href{https://github.com/r-hub/r-minimal}{\texttt{r-minimal}} takes this
to the extreme with compressed image size of 20MB and bespoke install
scripts, e.g., to cleverly remove compilers after package installation.
For R though, this advantage quickly deteriorates if more features are
needed, e.g., if adding compilers, packages, or especially if installing
powerful data science and communication tools, like RStudio or LaTeX.
Furthermore the distribution uses the
\href{https://www.musl-libc.org/}{\texttt{musl} C library} instead of
the \texttt{glibc} used in Debian/Ubuntu, for which more tooling and
experiences
exist\footnote{Cf. \href{https://github.com/rocker-org/rocker/issues/231}{https://github.com/rocker-org/rocker/issues/231}}.
In fact, using Debian's `slim' variant and removing capabilities, one
could probalby achieve similarly small image sizes on a more common
stack.

\textbf{\pkg{altRnative}}
(\url{https://github.com/ismailsunni/altRnative/}) is an experimental R
package for running the same code across multiple containerised versions
of R, intended for comparison across operating systems and different
implementations of R. It comes with a collection of \texttt{Dockerfile}s
and corresponding images in multiple combinations, currently including,
e.g., MRO, FastR, Fedora, and
TERR\footnote{See `Dockerfile`s at \href{https://github.com/ismailsunni/dockeRs}{https://github.com/ismailsunni/dockeRs}.}.

\hypertarget{data-science-images}{%
\subsection{Data Science images}\label{data-science-images}}

\label{datascience}

\emph{Data Science} is a widely discussed topic among all academic
disciplines \citep[e.g.,][]{donoho_50_2017}. The discussions shed a
light on the tools and craftspersonship behind the analysis of data with
computational methods. The practice of Data Science often involves a
combination tools and software stacks and requires a cross-cutting
skillset. This complexity and an inherent concern for openness and
reproducibility in the Data Science community lead to Docker being used
widely. This section presents exemplary Docker images and image stacks
featuring R intended for Data Science.

The \href{https://github.com/jupyter/docker-stacks/}{\textbf{Jupyter
Docker Stacks}} project are a set of ready-to-run Docker images
containing Jupyter applications and interactive computing tools
\citep{project_jupyter_jupyter_2018}. The \texttt{jupyter/r-notebook}
image includes R and ``popular packages'', and naturally also the
IRKernel (\url{https://irkernel.github.io/}), an R kernel for Jupyter so
that Jupyter Notebooks can contain R code cells. R is also included in
the catchall \texttt{jupyter/datascience-notebook}
image\footnote{\href{https://jupyter-docker-stacks.readthedocs.io/en/latest/using/selecting.html}{https://jupyter-docker-stacks.readthedocs.io/en/latest/using/selecting.html}}.
For example, these images allow users to quickly start a Jupyter
Notebook server locally or build their own specialised images on top of
stable toolsets. R is installed using the Conda package manager, which
can manage environments for various programming languages, pinning both
the R version and the versions of R
packages\footnote{See \texttt{jupyter/datascience-notebook}'s \texttt{Dockerfile} at \href{https://github.com/jupyter/docker-stacks/blob/master/datascience-notebook/Dockerfile\#L47}{https://github.com/jupyter/docker-stacks/blob/master/datascience-notebook/Dockerfile\#L47}.}.

\textbf{Kaggle} provides the
\href{https://hub.docker.com/r/kaggle/rstats}{\texttt{gcr.io/kaggle-images/rstats}}
image (previously \texttt{kaggle/rstats}) and
\href{https://github.com/Kaggle/docker-rstats}{corresponding
\texttt{Dockerfile}} for usage in their Machine Learning competitions
and easy access to the associated datasets. It includes machine learning
libraries such as Tensorflow and Keras (see also image
\texttt{rocker/ml} in Section~\nameref{rocker-gpu}), and also configures
the \CRANpkg{reticulate} package. The image uses a base image with
\emph{all packages from CRAN}, \texttt{gcr.io/kaggle-images/rcran},
which requires a Google Cloud Build as Docker~Hub would time
out\footnote{Originally, a stacked collection of over 20 images with automated builds on Docker~Hub was used, see \href{https://web.archive.org/web/20190606043353/http://blog.kaggle.com/2016/02/05/how-to-get-started-with-data-science-in-containers/}{https://web.archive.org/web/20190606043353/http://blog.kaggle.com/2016/02/05/how-to-get-started-with-data-science-in-containers/} and \href{https://hub.docker.com/r/kaggle/rcran/dockerfile}{https://hub.docker.com/r/kaggle/rcran/dockerfile}}.
The final extracted image size is over 25GB, which makes it debatable if
having everything available is actually convenient.

As a further example,
\href{https://radiant-rstats.github.io/docs/}{\textbf{Radiant} project}
provides several images, e.g.,
\href{https://hub.docker.com/r/vnijs/rsm-msba-spark}{\texttt{vnijs/rsm-msba-spark}},
for their browser-based business analytics interface based on
\CRANpkg{Shiny} (\texttt{Dockerfile}
\href{https://github.com/radiant-rstats/docker}{on GitHub}) and for use
in education as part of an MSc course. As data science often applies a
multitude of tools, this image favours inclusion over selection and
features Python, Postgres, JupyterLab and Visual Studio Code besides R
and RStudio, bringing the image size up to 9GB.

\textbf{Gigantum} (\url{http://gigantum.com/} is a platform for open and
decentralized data science with a focus on using automation and
user-friendly tools for easy sharing of reproducible computational
workflows. Gigantum builds on the \emph{Gigantum Client} (running either
locally or on a remote server) for development and execution of
data-focused \emph{Projects}, which can be stored and shared via the
\emph{Gigantum Hub} or via a zipfile export. The Client is a user
friendly interface to a backend using Docker containers to package,
build, and run Gigantum projects. It is configured to use a default set
of Docker base images (\url{https://github.com/gigantum/base-images}),
and users are able to define and configure their own custom images. The
available images include two with R based on Ubuntu Linux have the
\href{https://launchpad.net/~marutter/+archive/ubuntu/c2d4u3.5/}{\texttt{c2d4u}
CRAN PPA} pre-configured for installation of binary R
packages\footnote{\href{https://docs.gigantum.com/docs/using-r}{https://docs.gigantum.com/docs/using-r}}.
The R images vary in the included authoring environment, i.e., Jupyter
in \texttt{r-tidyverse} or both Jupyter \& RStudio in
\texttt{rstudio-server}. The independent image stack can be traced back
to the Gigantum environment and its features. The R images are based on
Gigantum's \texttt{python3-minimal} image, originally to keep the
existing front-end configuration, but also to provide a consistent
Python-to-R interoperability. The \texttt{Dockerfile}s also use build
args to specify bases, for example for different version of NVIDIA CUDA
for GPU
processing\footnote{See \href{https://github.com/gigantum/base-images/blob/master/_templates/python3-minimal-template/Dockerfile}{https://github.com/gigantum/base-images/blob/master/\_templates/python3-minimal-template/Dockerfile} for the \texttt{Dockerfile} of \texttt{python3-minimal}.}
so that appropriate GPU drivers can be enabled automatically when
supported. Furthermore, Gigantum's focus lies on environment management
via GUI and ensuring a smooth user interaction, e.g., with reliable and
easy conflict detection and resolution. For this reason, project
repositories store authoritative package information in a separate file
per package, allowing Git to directly detect conflicts and changes. A
Dockerfile is generated from this description that inherits from the
specified base image, and additional custom Docker instructions may be
appended by users, though Gigantum's default base images do not
currently include the \texttt{littler} tool, which is used by Rocker to
install packages within \texttt{Dockerfile}s. Because of these
specifics, instructions from \texttt{rocker/r-ubuntu} could \emph{not}
be readily re-used in this image stack (see
Section~\nameref{conclusions}). Both approaches enable \texttt{apt} as
an installation method, and this is exposed via the GUI-based
environment
management,\footnote{See \href{https://docs.gigantum.com/docs/environment-management}{https://docs.gigantum.com/docs/environment-management}}.
The image build and publication process is scripted with Python and JSON
template configuration files, unlike Rocker which relies on plain
\texttt{Dockerfile}s. A minor reason in the inception of the images were
also project constraints requiring a Rocker-incompatible licensing of
the \texttt{Dockerfile}s, i.e., the MIT License.

\textbf{Whole~Tale} (\url{https://wholetale.org}) is a National Science
Foundation (NSF) funded project developing a scalable, open-source,
multi-user platform for reproducible research
\citep{brinckman2019, chard2019a}. A central goal of the platform is to
enable researchers to easily create and publish executable research
objects\textbackslash{}footnote\{In Whole\textasciitilde{}Tale a
\emph{tale} is a research object that contains metadata, data (by copy
or reference), code, narrative, documentation, provenance, and
information about the computational environment to support computational
reproducibility\} associated with published research \citep{chard2019b}.
Using Whole~Tale, researchers can create and publish Rocker-based
reproducible research objects to a growing number of repositories
including DataONE member nodes, Zenodo and soon Dataverse. Additionally,
Whole~Tale supports automatic data citation and is working on
capabilities for image preservation and provenance capture to improve
the transparency of published computational research artifacts
\citep{mecum2018, mcphillips2019}. For R users, Whole~Tale attempts to
combine the strengths of the Rocker Project's curated Docker images with
the Jupyter Project's
\texttt{repo2docker}\footnote{See \href{https://repo2docker.readthedocs.io/}{https://repo2docker.readthedocs.io/}}
component to simplify the customization of R-based environments for
researchers with limited experience with either Docker or Git. Multiple
options have been discussed to allow users to change the base image used
in \texttt{repo2docker} from the default Ubuntu LTS (long-term support)
required to support the Rocker Project images. Whole~Tale implemented a
custom
\texttt{RockerBuildPack}\footnote{See \href{https://github.com/whole-tale/repo2docker\_wholetale}{https://github.com/whole-tale/repo2docker\_wholetale}}
to support customization of the \texttt{rocker/geospatial} image through
\texttt{repo2docker}
composability\footnote{Composability refers to the ability to combine multiple package managers -- such as R, `pip`, and `conda`}.
This works in part because Rocker images are based on a Debian
distribution, so the instructions created by \texttt{repo2docker} for
Ubuntu work because of compatible shell and package manager.

\hypertarget{use-cases-and-applications}{%
\section{Use cases and applications}\label{use-cases-and-applications}}

\label{applications}

\hypertarget{interfaces-for-docker-in-r}{%
\subsection{Interfaces for Docker in
R}\label{interfaces-for-docker-in-r}}

\label{interfaces}

Interfacing with the Docker daemon is typically done through the
\href{https://docs.docker.com/engine/reference/commandline/cli/}{Docker
Command Line Interface} (Docker CLI). However, moving back and forth
between an R console and the command line can create friction in
workflows and reduce reproducibility. A number of first-order R packages
provide a interface to the Docker CLI, allowing to automate interaction
with the Docker CLI from an R console.

Each of these packages has particular advantages as they provide
function wrappers for interacting with the Docker CLI at different
stages of a container's life cycle. Examples of such interactions are
installing the Docker software, creating \texttt{Dockerfile}s
(\CRANpkg{dockerfiler}, \pkg{containerit}), building images and
launching a containers (\CRANpkg{stevedore}, \pkg{docker}) on a local
machine or on the cloud. As such, the choice of which package is most
useful depends on the use-case at hand, but also the users level of
expertise.

\begin{tabular}{l|l|l|l|l|l|l}
\hline
\rotatebox{-90}{Functionality} & \rotatebox{-90}{AzureContainers} & \rotatebox{-90}{babelwhale} & \rotatebox{-90}{dockermachine} & \rotatebox{-90}{dockyard} & \rotatebox{-90}{harbor} & \rotatebox{-90}{stevedore}\\
\hline
Generate a Dockerfile &  &  &  & \checkmark &  & \\
\hline
Build an image & \checkmark &  &  & \checkmark &  & \\
\hline
Execute a container locally or remotely & \checkmark & \checkmark & \checkmark & \checkmark & \checkmark & \checkmark\\
\hline
Deploy or manage an instances in the cloud & \checkmark &  & \checkmark &  & \checkmark & \checkmark\\
\hline
Interact with an instance (e.g., file transfer) &  & \checkmark & \checkmark &  &  & \checkmark\\
\hline
Manage storage of images &  &  &  &  & \checkmark & \checkmark\\
\hline
Supports Docker and Singularity &  & \checkmark &  &  &  & \\
\hline
Direct access to Docker API instead of using the CLI &  &  &  &  &  & \checkmark\\
\hline
Installing Docker software &  &  & \checkmark &  &  & \\
\hline
\end{tabular}

\textbf{\pkg{harbor}} (\url{https://github.com/wch/harbor} is not
actively maintained anymore, but should be honorably mentioned as the
first R package for managing Docker images and containers. It uses the
\CRANpkg{sys} package to run system commands against the Docker CLI,
both locally and through an SSH connection, and has convenience
functions, e.g., for listing and removing containers/images and for
accessing logs. The output of container executions are converted to
appropriate R types. The Docker CLI's basic functionality, while
evolving quickly and with small concern for avoiding breaking changes,
is unchanged for a long time so a core function such as
\texttt{harbor::docker\_run(image\ =\ "hello-world")} still works
depsite the stopped development.

\textbf{\CRANpkg{stevedore}}
(\url{https://cran.r-project.org/package=stevedore}) is currently the
most powerful Docker client in R. It interfaces with the Docker daemon
over the Docker HTTP
API\footnote{\href{https://docs.docker.com/engine/api/latest/}{https://docs.docker.com/engine/api/latest/}}
via a Unix socket on Linux or MacOS, over a named pipe on Windows, or
over an HTTP/TCP connection. The package is the only one not using
system calls to the \texttt{docker} CLI for managing images and
containers and easily exposes connections to remote Docker daemons,
which has to be configured on the Docker level otherwise. Using the API
gives access to more information and is system independent and likely
more reliable than parsing command line output. \CRANpkg{stevedore}'s
own interface is automatically generated based on the OpenAPI
specification of the Docker daemon, but still similar to the Docker CLI.
The interface is similar to R6 objects, in that a
\texttt{stevedore\_object} has a number of functions attached to it that
can be called, and multiple specific versions of the Docker API can be
supported thanks to the automatic
generation\footnote{See \href{https://github.com/richfitz/stevedore/tree/master/inst/spec}{https://github.com/richfitz/stevedore/tree/master/inst/spec} and \href{https://github.com/richfitz/stevedore/blob/master/development.md}{https://github.com/richfitz/stevedore/blob/master/development.md}.}.

\textbf{\CRANpkg{AzureContainers}} is an interface to a number of
container-related services in Microsoft's
\href{https://azure.microsoft.com/}{Azure Cloud}, see
Section~\nameref{azurecontainers}. While it is mainly intended for
working with Azure, as a convenience feature it includes lightweight,
cross-platform shells to Docker and Kubernetes (tools \texttt{kubectl}
and \texttt{helm}). These can be used to create and manage arbitrary
Docker images and containers, as well as Kubernetes clusters on any
platform or cloud service.

\textbf{\CRANpkg{babelwhale}} allows executing and interacting with
containers, which can use either Docker or Singularity as a backend
\citep{cannoodt_babelwhale_2019}. The package provides a unified
interface to interact with Docker and Singularity containers. Users can,
for example, execute a command inside a container, mount a volume or
copy a file.

\textbf{\pkg{dockyard}}
(\url{https://github.com/thebioengineer/dockyard}) has the goal of
lowering barrier to creating \texttt{Dockerfile}s, building Docker
images, and deploying Docker containers. The package follows the
increasingly used piping paradigm of the \texttt{tidyverse} style of
programming for chaining R functions representing the instructions in a
\texttt{Dockerfile}. An existing \texttt{Dockerfile} can be used as a
template. \pkg{dockyard} also includes wrappers for common steps, such
as installing an R package or copying files, and build-in functions for
building an image running a container, to make using Docker even more
approachable to R users with a single API.

\textbf{\pkg{dockermachine}}
(\url{https://github.com/cboettig/dockermachine}) is an R package to
provide a convenient interface to
\href{https://docs.docker.com/machine/overview/}{Docker~Machine} from R.
The CLI tool \texttt{docker-machine} allows users to create and manage
virtual host on local computers, local data centers, or at cloud
providers. A local Docker installation can be configured to transparntly
forward all commands issued on the local Docker CLI to a selected
(remote) virtual host. Docker~Machine was especially crucial for local
use in early days of Docker, when no native support was available for
Mac or Windows computers, but remains relevant for provisioning on
remote systems. The package has not received any updates for two years,
but is functional with a current version of \texttt{docker-machine}
(\texttt{0.16.2}). It potentially lowers the barriers for R users to run
containers on various hosts, if using the Docker~Machine CLI directly is
perceived as a barrier, or enables scripted workflows with remote
processing.

\hypertarget{capture-and-create-environments}{%
\subsection{Capture and create
environments}\label{capture-and-create-environments}}

\label{envs}

Several second order R packages attempt to make the process of creating
Docker images and using containers for specific tasks, such as running
tests or rendering reproducible reports, easier. While authoring and
managing an environment with Docker by hand is possible and feasible for
experts\footnote{See, e.g., this tutorial by RStudio on how to manage environments and package versions and to ensure deterministic image builds with Docker: \href{https://environments.rstudio.com/docker}{https://environments.rstudio.com/docker}.},
the following examples show the power of automation when environments
become too cumbersome or acquiring the skills is not possible in due
course. Especially \emph{version pinning}, with packages \pkg{remotes}
and \pkg{versions} for R or by using MRAN, and with system package
managers for different operating systems, can greatly increase the
reproducibility of built images and are common approaches.

\textbf{\pkg{dockerfiler}}
(\url{https://github.com/ColinFay/dockerfiler/}) is an R package
designed for building \texttt{Dockerfile}s straight from R. A scripted
creation of a \texttt{Dockerfile} enables iteration and automation, for
example for packaging applications for deployment (see
\nameref{deployment}). Being scriptable from R developers can leverage
the tools available in R to parse a \texttt{DESCRIPTION} file, to get
system requirements, to list dependencies, versions, etc.
\textbf{\pkg{containerit}}
(\url{https://github.com/o2r-project/containerit/}) attempts to take
this one step further and includes these tools to automatically create a
\texttt{Dockerfile} that can execute a given workflow
\citep{nust_containerit_2019}. \pkg{containerit} accepts R
\texttt{sessionInfo} objects as input and provides helper functions to
derive these from workflows, e.g., an R script or R Markdown document,
by analysing the session state at the end of the workflow. It relies on
the \pkg{sysreqs} (\url{https://github.com/r-hub/sysreqs/}) package and
it's mapping of package system dependencies to platform specific
installation package
names\footnote{See \href{https://sysreqs.r-hub.io/}{https://sysreqs.r-hub.io/}.}.
\pkg{containerit} uses \CRANpkg{stevedore} to streamline the user
interaction and improve the created \texttt{Dockerfile}s, e.g., by
running a container for the desired base image to extract the already
available R packages. \textbf{\pkg{dockr}}
(\url{https://github.com/smaakage85/dockr}) is a very similar package
attempting to mirror a given R session, including local non-CRAN
packages\citep{kjeldgaard_dockr_2019}. Users can manually add statements
for non-R dependencies to the \texttt{Dockerfile}.
\textbf{\CRANpkg{liftr}} aims to solve the problem of persistent
reproducible reporting in statistical computing based on the R Markdown
format \citep{xie2018} for dynamic documents
\citep[\href{https://nanx.me/liftr/}{https://nanx.me/liftr/}, ][]{liftr2019}.
The irreproducibility of authoring environments can become an issue for
collaborative documents and large-scale platforms for processing
documents. \CRANpkg{liftr} makes the document the main and sole workflow
control file and the only file that needs to be shared between
collaborators for consistent environments, e.g.~demonstrated in the
DockFlow project (\url{https://dockflow.org}). It introduces new fields
to the R Markdown document header, allowing users to manually declare
the dependencies, including versions, for rendering the document. The
package then generates a \texttt{Dockerfile} from this metadata and
provides a utitility function to render the document inside a Docker
container, i.e., \texttt{render\_docker("foo.Rmd")}. An RStudio addin
even allows compilation of documents with a single push of a button.

The package \textbf{\CRANpkg{renv}}
(\url{https://rstudio.github.io/renv/}) helps users to manage the state
of the R library in a reproducible way, further providing isolation and
portability \citep{renv2019}. The package does not cover system
dependencies, though, the \CRANpkg{renv}-based environment can be
transferred into a container either by restoring the environment based
on the main configuration file \texttt{renv.lock} or by storing the
\CRANpkg{renv}-cache on the host and not in the container
\citep{ushey_using_2019}.

\hypertarget{using-r-to-power-enterprise-software-in-production-environments}{%
\subsection{Using R to power enterprise software in production
environments}\label{using-r-to-power-enterprise-software-in-production-environments}}

\label{enterprise}

R has been historically viewed as a tool for analysis and scientific
research, but not for creating software that corporations can rely on
for production services. However, thanks to advancements in R running as
a web service, along with with the ability to deploy R in Docker
containers, modern enterprises are now capable of having real-time
machine learning powered by R. A number of packages and projects enabled
R to respond to client reqests over TCP/IP and local socket servers,
such as \CRANpkg{Rserve}, \CRANpkg{svSocket},
\href{http://www.rapache.net}{rApache} and more recently
\CRANpkg{plumber} (\url{https://www.rplumber.io/}) and \pkg{RestRserve}
(\url{http://restrserve.org}), which even processes incoming requests in
parallel with forked processes using \CRANpkg{Rserve}. The latter two
also provide documentation for deployment with Docker or ready to use
automated builds of
images\footnote{See \href{https://www.rplumber.io/docs/hosting.html\#docker}{https://www.rplumber.io/docs/hosting.html\#docker}, \href{https://hub.docker.com/r/trestletech/plumber/}{https://hub.docker.com/r/trestletech/plumber/} and \href{https://hub.docker.com/r/rexyai/restrserve/}{https://hub.docker.com/r/rexyai/restrserve/}.}.
These software allow other (remote) processes and programming languages
to interact with R and to expose R-based function in a service
architecture with HTTP APIs. APIs based on these package can be deployed
with scalability and high availability using containers. This pattern of
deploying code matches those used by software engineering services
created in more established languages in the enterprise domain, such as
Java or Python, and R can be used alongside those langages as a first
class member of a software engineering technical stack.

CARD.com implemented a web application for the optimization of the
acquisition flow and the real-time analysis of debit card transactions.
The software used \CRANpkg{Rserve} and rApache and was deployed in
Docker containers. The R session behind \CRANpkg{Rserve} acted as a
read-only in-memory database, which was extremely fast and scalabale,
for the many concurrent rApache processes responding to the live-scoring
requests of various divisions of the company. Similarly dockerized R
scripts were responsible for the ETL processes and even the
client-facing email, text message and push notification alerts sent in
real-time based on card transactions. The related Docker images were
made available at \url{https://github.com/cardcorp/card-rocker}. The
images extend \texttt{rocker/r-base} and additionally entailed an SSH
client and a workaround for being able to mount SSH keys from the host,
Pandoc, the Amazon Web Services (AWS) SDK, and Java, which is required
by the AWS SDK. The AWS SDK allowed to run R consumers reading from
real-time data processing streams of
\href{https://aws.amazon.com/kinesis/}{AWS
Kinesis}\textbackslash{}footnote\{See useR!2017 talk \emph{Stream
processing with R in AWS} at
\textbackslash{}href\{\url{https://static.sched.com/hosted_files/user2017/2f/AWR\%20Kinesis\%20at\%20useR\%202017.pdf}\}\{\url{https://static.sched.com/hosted_files/user2017/2f/AWR}
Kinesis at use 02017.pdf\}.\}. The applications were deployed on Amazon
Elastic Container Service (\href{https://aws.amazon.com/ecs/}{ECS}). The
main learnings from using R in Docker was the imporance of not only
pinning the R package versions via MRAN, but also moving away from
Debian testing to a distribution with long-term support. For the use
case at hand, this switch served the priority to have more control over
upstream updates, and to minimize the risk of breaking the automated
builds of the Docker images and production jobs.

The AI @ T-Mobile team created a set of neural network machine learning
natural language processing models to help customer care agents manage
text-based messages for customers \citep{t-mobile_enterprise_2018}. For
example, one model quickly identifies if a message is from a customer or
not
\citep[see \CRANpkg{Shiny}-based \href{https://secure.message.t-mobile.com/v1/shiny/is-customer/app/}{demo}, ][]{nolis_small_2019},
others tell which customers are likely to make a repeat purchase. If a
data scientist creates a machine learning model and exposes it through a
\CRANpkg{plumber} API, then someone else on the marketing team could
write software that sends different emails depending on that real-time
prediction. The models are convolutional neural networks that use the
\CRANpkg{keras} package and run in a Rocker Docker image. The
corresponding \texttt{Dockerfile}s are published
\href{https://github.com/tmobile/r-tensorflow-api}{on GitHub}. Since the
models power tools for agents and customers, they need to have extremely
high uptime and reliability. The AI @ T-Mobile team found that the
models performed well and today these models power real-time services
that are called over a million times a day.

\hypertarget{continuous-integration-continuous-delivery-and-deployment}{%
\subsection{Continuous integration, continuous delivery, and
deployment}\label{continuous-integration-continuous-delivery-and-deployment}}

\label{cicd} \label{deployment}

The cloud is the natural environment of containers, and subsequently
containers are the go-to mechanism to deploy R server applications. More
and more continuous integration (CI) and continuous delivery (CD)
services also use containers, opening up new options for use. The
controlled nature of containers, i.e., the possibility to define the
software environment very well, even on remote machines, is crucial
here, for example to match test or build environments with production
environments.

As a first example, \CRANpkg{Shiny} is the most popular package for
creating interactive online dashboards with R and it enables users with
very diverse backgrounds to create stable and user friendly web
applications. The ShinyProxy (\url{https://www.shinyproxy.io/}) is an
open source tool to deploy Shiny apps in an enterprise context. They
feature single sign-on, but also in scientific use cases
\citep[e.g., ][]{savini_epiexplorer_2019,glouzon_structurexplor_2017}.
ShinyProxy uses Docker containers to isolate user sessions and to
achieve scalability for multi-user scenarious with multiple apps.
ShinyProxy itself is written in Java to accomodate corporate
requirements and may itself run in a container for stability and
availability. The tool is build on ContainerProxy
(\url{https://www.containerproxy.io/}), which provides similar features
for executign long-running R jobs or interactive R sessions. The started
containers can run on a regular Docker host but also in clusters.

Second, containers can be used to create \textbf{platform installation
packages} in a DevOps setting. The
\href{https://www.opencpu.org/}{OpenCPU} system provides an HTTP API for
data analysis based on R. \citet{ooms_opencpu_2017} describes how
various platform-specific installation files for OpenCPU are created
using Docker~Hub: the automated builds install the software stack from
source on different operating systems; afterwards a script file
downloads the images and extracts the OpenCPU binaries.

\begin{itemize}
\tightlist
\item
  \href{https://github.com/dynverse/dynwrap_containers/blob/master/.travis.yml}{dynwrap}
  \citep{rcannood}

  \begin{itemize}
  \tightlist
  \item
    For this project, we use travis-ci to build rocker-derived
    containers, test them, and only push them to Docker~Hub (from
    travis-ci.org) if the integration tests succeed.
  \end{itemize}
\end{itemize}

For example, the
\textbf{\href{https://github.com/ThinkR-open/golem}{\texttt{golem}}}
package makes an heavy use of \texttt{dockerfiler} when it comes to
creating the \texttt{Dockerfile} for building production-grade Shiny
applications and deploying them.

\begin{itemize}
\tightlist
\item
  \texttt{RSelenium}
\item
  \href{https://cloudyr.github.io/googleComputeEngineR/}{\texttt{googleComputeEngineR}}
  (function \texttt{gce\_vm\_template()})

  \begin{itemize}
  \tightlist
  \item
    To enable quick deployments of key R services such as RStudio and
    Shiny onto cloud virtual machines (VMs), this package utilises
    Dockerfiles to move the labour of setting up those services from the
    user to a premade Docker image. For example, by specifying the
    template \texttt{template="rstudio"} in
    \texttt{gce\_vm\_template()/gce\_vm()} an up to date RStudio Service
    image is launched. Specifying \texttt{template="rstudio-gpu"} will
    launch an RStudio Server image with a GPU attached, etc.\\
  \end{itemize}
\item
  \href{https://github.com/sckott/analogsea}{\texttt{analogsea}}
  (digital ocean R client)
\item
  \href{https://www.rplumber.io/docs/hosting.html\#docker}{\texttt{plumber}}
\item
  \href{https://github.com/tmobile/r-tensorflow-api}{production
  deployment of neural networks} \citep[\citet{nolistic}]{jnolis}
\item
  Kubernetes - A popular platform for managing Docker containers is
  \href{https://kubernetes.io/}{Kubernetes}, which is used for a wide
  variety of containerized applications. It may be your organisation has
  a Kubernetes cluster already for other applications. Docker containers
  are used within Kubernetes clusters to hold native code, for which
  Kubernetes creates a framework around network connections and scaling
  of resources up and down. An introduction on using
  \href{https://code.markedmondson.me/r-on-kubernetes-serverless-shiny-r-apis-and-scheduled-scripts/}{Kubernetes
  with R is published at this blog post}.
\end{itemize}

\label{azurecontainers} \textbf{\CRANpkg{AzureContainers}} is an
umbrella package providing interfaces three commercial services of
Microsoft's Azure Cloud, namely
\href{https://azure.microsoft.com/en-us/services/container-instances/}{Container
Instances} for running individual containers,
\href{https://azure.microsoft.com/en-us/services/container-registry/}{Container
Registry} for private image distribution, and
\href{https://azure.microsoft.com/en-us/services/kubernetes-service/}{Kubernetes
Service} for orchestrated deployments. While a package like
\CRANpkg{plumber} provides the infrastructure for turning an R workflow
a service, for production purposes it is usually necessary to take into
account scalability, reliability and ease of management. A
\emph{cluster} of containers, orchestrated as a single deployment, e.g.,
with \href{https://kubernetes.io}{Kubernetes}, can mitigate limitations
on request volumes or a container occupied with computationally
intensive task. A cluster features load-balancing, autoscaling of
containers across numerous servers (in the cloud or on premise), and
restarting failed ones. AzureContainers provides an R-based interface to
these features for Azure. On the client side, it provides simple
wrappers to Docker and to Kubernetes' management tools \texttt{kubectl}
and \texttt{helm}. The package simplifies complex infrastructure
management to a number of R function calls, given an Azure account with
sufficient
credit\footnote{See \emph{"Deploying a prediction service with Plumber"} vignette for details:  \href{https://cran.r-project.org/web/packages/AzureContainers/vignettes/vig01_plumber_deploy.html}{https://cran.r-project.org/web/packages/AzureContainers/vignettes/vig01\_plumber\_deploy.html}.}.

The prevelance of Docker in industry naturally leads to usage of
containers with R in such settings as well, as customers already manage
platforms in Docker containers. These products often entail a high
amount of open source software in combination with proprietary layers
addign the relevant commercializable features. One such example is
RStudio's data science platform
\href{https://rstudio.com/products/team/}{RStudio Team}. It allows teams
of data scientists and their respective IT/DevOps groups to develop and
deploy code in R and Python around the RStudio Open Source Server inside
of Docker images, without requiring users to learn new tools or directly
interact with containers. The best practices for
\href{https://support.rstudio.com/hc/en-us/articles/360021594513-Running-RStudio-with-Docker-containers}{running
RStudio with Docker containers} as well as
\href{https://github.com/rstudio/rstudio-docker-products}{Docker images}
for RStudio's commerical products are openly available.

\hypertarget{common-or-public-work-environments}{%
\subsection{Common or public work
environments}\label{common-or-public-work-environments}}

\label{workenvs}

The fact that Docker images are portable and well defined make them
useful when more than one person needs access to the same computing
environment. This is even more useful when some of the users do not have
the expertise to create such an environment themselves, and when these
environments can be run in public or shared infrastructure.

\textbf{\pkg{holepunch}} (\url{https://github.com/karthik/holepunch}) is
an R package that was designed to make sharing work environments
accessible to novice R users based on \textbf{Binder}. The
\href{https://mybinder.readthedocs.io/en/latest/}{Binder project},
maintained by the team behind Jupyter, makes it possible for users to
create and share their computing environments with others
\citep{jupyter_binder_2018}. A \emph{BinderHub} allows anyone with
access to a web browser and an internet connection to launch a temporary
instance of these custom environments and execute any workflows
contained within. From a reproducibility standpoint, Binder makes it
exceedingly easy to compile a paper, visualize data, and run small
examples from papers or tutorials without the need for any local
installation. To set up Binder for a project, a user typically starts at
an instance of a BinderHub and passes the location of a repository with
a workspace, e.g., a hosted Git repository, or a data repository like
Zenodo. Binder's core internal tool is
\href{https://repo2docker.readthedocs.io/en/latest/config_files.html}{\texttt{repo2docker}}.
It deterministically builds a Docker image by parsing the contents of a
repository, e.g., project dependency configurations or simple
configuration files. In the most powerful case, \texttt{repo2docker}
builds a given \texttt{Dockerfile}. While this approach works well for
most run of the mill Python projects, it is not so seamless for R
projects. For any R projects that use the Tidyverse suite
\citep{wickham_welcome_2019}, the time and resources required to build
all dependencies from source can often time out before completion,
making it frustrating for the average R user. \pkg{holepunch} removes
some of these limitations by leveraging Rocker images that contain the
Tidyverse along special Jupyter dependencies, and only installs
additional packages from CRAN and Bioconductor that are not already part
of these images. It short cicuits the configuration file parsing in
\texttt{repo2docker} and starts with the Binder/Tidyverse base images,
which eliminates a large part of the build time and in most cases
results in a binder instance launching within a minute. \pkg{holepunch}
as a side effect also creates a \texttt{DESCRIPTION} file which then
turns any project into a research compendium
\citep{marwick_packaging_2018}. The \texttt{Dockerfile} included with
the project can also be used to launch a RStudio server locally, i.e.,
independent of Binder, which is especially useful when more or special
computational resources can be provided there. The local image usage
reduces the number of seperately managed environments and thereby
reduces work and increases portability and reproducibility.

In \textbf{high-performance computing}, one use for containers is to run
workflows on shared local hardware where teams manage their own
high-performance servers. This can follow one of several design
patterns: users may deploy containers to hardware as a work environment
for a specific project, conatiners may provide per-user persistent
environments, or a single container can act as a common multi-user
environment for a server. In all cases, though, the containerized
approach provides several advantages: First, users may use the same
image and thus work environment on desktop and laptop computers, as
well. The former models provide modularity, while the latter approach is
most similar to a simple shared server. Second, software updates can be
achieved by updating and redeploying the container, rather than tracking
local installs on each server. Third, the containerized environment can
be quickly deployed to other hardware, cloud or local, if more resources
are neccessary or in case of server destruction or failure. In any of
these cases, users need a method to interact with the containers, be it
and IDE, or command-like access and tools such as SSH, which is usually
not part of standard container recipes and must be added. The
Rocker~Project provides containers pre-installed with the RStudio~IDE.
In cases where users store nontrivial amounts of data for their
projects, data needs to persist beyond the life of the container. This
may be via in shared disks, attached network volumes, or in separate
storage where it is uploaded between sessions. In the case of shared
disks or network-attached volumes, care must be taken to persist user
permissions, and of course backups are still neccessary. When working
with multiple servers, an automation framework such as
\href{https://www.ansible.com}{Ansible} may be useful for managing
users, permisions, and disks along with containers.

\label{rocker-gpu} Using \textbf{GPUs} (graphical processing units) as a
specialised hardware from containerized common work environments is also
possible and useful \citep{haydel_enhancing_2015}. GPUs are increasingly
popular for compute-intensive machine learning (ML) tasks, e.g., deep
artificial neural networks \citep{schmidhuber_deep_2015}. Though in this
case, containers are not completely portable between hardware
environments, but the software stack for ML with GPUs is so complex to
set up that a ready-to-use container is helpful. Containers running GPU
software require drivers and libraries specific to GPU models and
versions, and containers require a specialized runtime to connect to the
underlying GPU hardware. For NVIDIA GPUs, the
\href{https://github.com/NVIDIA/nvidia-docker}{NVIDIA Container Toolkit}
includes a specialized runtime plugin for Docker and a set of base
images with appropriate drivers and libraries. The Rocker~Project
\href{https://github.com/rocker-org/ml}{has a repository} with (beta)
images based on these that include GPU-enabled versions of
machine-learning R packages, e.g., \texttt{rocker/ml} and
\texttt{rocker/tensorflow-gpu}.

Teaching is a further example where shared computing environments and
sandboxing can greatly improve the process. First, \textbf{prepared
environments for teaching} are especially helpful for courses that
require access to a relatively complex setup of software tools, e.g.,
database systems. R is very useful tool for interfacing with databases,
because almost every open source and proprietary database system has an
R package that allows users to connect and interact with the it. This
flexibility is even broadened by \CRANpkg{DBI}, which allows to create a
common API for interfacing these databases, or the \CRANpkg{dbplyr}
package, which runs \CRANpkg{dplyr} code straight against the database
as queries. But learning and teaching these tools comes with the cost of
deploying or having access to an environment with the software and
drivers installed. For people teaching R, it can become a barrier if
they need to install local versions of database drivers, or to connect
to remote instances which might or might not be made available by IT
services. Giving access to a sandbox for the most common database
environments is the idea behind
\href{https://github.com/ColinFay/r-db}{\texttt{r-db}}, a Docker image
that contains everything needed to connect to a database from R.
Notably, with \texttt{r-db}, the users don't have to install complex
drivers or to configure their machine in a specific way. The
\texttt{rocker/tidyverse} base image ensures that users can also readily
use packages for analysis, display, and reporting. Second, the idea of a
common environment and partitioning allow using \textbf{containers in
teaching for secure execution and automated testing} of submissions by
students. \href{https://dodona.ugent.be}{Dodona} is a web platform
developed at Ghent University and is used to teach students basic
programming skills, and it uses Docker containers to test submissions by
students. This means that both the code testing the students'
submissions and the submission itself are executed in a predictable
environment, avoiding compatibility issues between the wide variety of
configurations used by students. The containerization is also used to
shield the Dodona servers from bad or even malicious code: memory, time
and I/O~limits are used to make sure students can't overload the system.
The web application managing the containers communicates with them by
sending configuration information as a JSON document over standard
input. Every Dodona Docker image shares a \texttt{main.sh} file that
passes through this information to the actual testing framework, while
setting up some error handling. The testing process in the Docker
containers sends back the test results by writing a JSON document to its
standard output channel. In June 2019, R support was added to Dodona
using an image derived from the \texttt{rocker/r-base} image that sets
up the \texttt{runner} user and and \texttt{main.sh} file expected by
Dodona\footnote{\href{https://github.com/dodona-edu/docker-images/blob/master/dodona-r.dockerfile}{https://github.com/dodona-edu/docker-images/blob/master/dodona-r.dockerfile}}.
It also installs the packages required for the testing framework and the
exercises so that this doesn't have to happen every time a student's
submission is evaluated. The actual testing of R exercises is done using
a custom framework loosely based on \CRANpkg{testthat}. During the
development of the testing framework it was found that the
\CRANpkg{testthat} framework did not provide enough information to its
reporter system to send back all the fields required by Dodona to render
its feedback. Right now, multiple statistics courses are developing
exercises to automate the feedback for their lab classes.

\textbf{RCloud} (\url{https://rcloud.social}) is a cloud-based platform
for data analysis, visualisation and collaboration using R. It provides
a \texttt{rocker/drd} base image for easy evaluation of the
platform\footnote{\href{https://github.com/att/rcloud/tree/master/docker}{https://github.com/att/rcloud/tree/master/docker}}.

\hypertarget{processing}{%
\subsection{Processing}\label{processing}}

\label{processing}

The portability of containerized environments becomes particularly
useful for improving expensive processing of data or shipping complex
processing pipelines.

First, it is possible to \textbf{offload complex processing to a
server}. \ldots{}

Second, when processes can be \textbf{run in parallel} for speeding up R
code.

For example,
\href{https://CRAN.R-project.org/package=googleComputeEngineR}{\texttt{googleComputeEngineR}}'s
\texttt{gce\_vm\_cluster()} function can create clusters of 2 or more
virtual machines, running multi-CPU architectures. Instead of running a
local R script with the local CPU and RAM restrictions, the same code
can be processed on all CPU threads of the cluster of machines in the
cloud, all running in a Docker container with the same R environments.
This is achieved through \texttt{googleComputeEngineR}'s integration
with the R parralisation library
\href{https://CRAN.R-project.org/package=future}{\texttt{library(future)}
by Henrik Bengtsson}. Local R computation can be thrown up to a
multi-CPU and VM environment to achieve parrell computation in a few
lines of R code in your local session. -
\href{https://cloudyr.github.io/googleComputeEngineR/articles/massive-parallel.html}{some
demonstrations are available here}. - \href{https://cloud.run}{Google
Cloud Run} - This is a CaaS (Containers as a Service) that lets you
launch a Docker container without worrying about underlying
infrastructure. This dispenses with the developer creating a cloud
server to run the Docker image on, by abstracting away those servers to
a more serverless configuration. Cloud Run lets you run your code on top
of a managed or your own Kubernetes cluster running Knative, and can
accept any Docker image. The service takes care of network ingress,
scaling machines up and down to zero, authentication and authorisation,
all features which are non-trivial for a developer to create on their
own. This can be used to scale up R code to millions of instances if
they need to, with little or no changes to existing code. An R
implementation is shown here at
\href{https://github.com/MarkEdmondson1234/cloudRunR}{cloudRunR} which
uses Cloud Run to create a scalable R plumber API.
\url{https://code.markedmondson.me/googleCloudRunner/index.html}

\href{https://cloud.google.com/cloud-build/}{Google Cloud Build} and the
Google Container Registry are a continuous intergation service
respectively image registry that offload building of images to the
cloud, while serving the needs of commercial environments that helps
move the workload of building the Docker images to an online service,
while Cloud Build can be set up to build the Dockerfiles on each GitHub
commit or release. This means you do not need to build the Docker images
locally, which can tie up resources since Docker images can be several
GBs and take a long time to compile. Google Cloud Build works alongside
to allow you to build private and public Docker images, which allows you
to build up your own dependency graphs for downstream applications. -
\url{https://code.markedmondson.me/googleCloudRunner/articles/cloudbuild.html}

\begin{itemize}
\tightlist
\item
  \CRANpkg{batchtools} \citep{Lang2017batchtools} can
  \href{https://mllg.github.io/batchtools/reference/makeClusterFunctionsDocker.html}{schedule
  jobs with Docker Swarm}
\item
  scalable deployments, e.g., start with numerous Shiny talks mentioning
  Rocker at useR!2017
\item
  \href{https://github.com/dynverse/dynmethods}{dynmethods}
  \citep{rcannood}: In order to evaluate ±50 computational methods which
  all used different environments (R, Python, C++, \ldots{}), we wrapped
  each of them in a docker container and can execute these methods from
  R. Again, all of these containers are being built on travis-ci, and
  will only be pushed to Docker~Hub if the integration test succeeds.
\item
  \CRANpkg{drake},
  \url{https://docs.ropensci.org/drake/index.html?q=docker\#with-docker}
\end{itemize}

In the RStudio Server Pro 1.2 release in 2019, RStudio added new
functionality called
\href{https://solutions.rstudio.com/launcher/overview/}{Launcher}. It
gives users the ability to spawn R sessions and background jobs in a
scalable way on external clusters, e.g.,
\href{https://support.rstudio.com/hc/en-us/articles/360019253393-Using-Docker-images-with-RStudio-Server-Pro-Launcher-and-Kubernetes}{Kubernetes
based on Docker images} or \href{https://slurm.schedmd.com/}{Slurm}
clusters, and optionally, with Singularity containers. A key benefit of
Launcher is the ability for R and Python users to interactively work in
RStudio without learning about Docker at all - while still leveraging
containers and Kubernetes. Users can submit batch jobs to Kubernetes
clusters without writing specific deployment scripts.

\label{pipelines} Third, containers are perfectly suited for
\textbf{packaging and executing software pipelines} and required data.
Containers allow building complex processing pipelines that are
independent from the host programming language. Due to its original use
case (see~\nameref{introduction}), Docker has no standard mechanisms for
chaining containers together; it lacks definitions and protocols for
environment variables, volume mounts and/or ports that could enable
transfer of input (parameters and data) and output (results) to and from
containers. Some packages, e.g., \pkg{containerit}, do provide Docker
images that can be used similarly to CLIs, but their usage is
cumbersome\footnote{\href{https://o2r.info/containerit/articles/container.html}{https://o2r.info/containerit/articles/container.html}}.
\textbf{\pkg{outsider}} (\url{https://docs.ropensci.org/outsider/})
tackles the problem of integrating external programs into an R workflow
\citep{bennett_outsider_2020}. Installation and usage of external
programs can be difficult, convoluted and even impossible if the
platform is incompatible. Therefore \pkg{outsider} uses the
platform-independent Docker images to encapsulate processes in
\emph{outsider modules}. Each outsider module has a \texttt{Dockerfile}
and an R package with functions for interacting with the encapsulated
tool. Using only R functions, an end-user can install a module with the
\pkg{outsider} package and then call module code to integrate a tool
into their own R-based workflow seamlessly. The \pkg{outsider} package
and module manage the containers and handle the transmission of
arguments and the transfer of files to and from a container. These
functionalities also allow a user to launch module code on a remote
machine via SSH, expanding the potential computational scale. Outsider
modules can be hosted code-sharing services, e.g., on GitHub, and
\pkg{outsider} contains discovery functions for them.

\hypertarget{packaging-research-reproducibly}{%
\subsection{Packaging research
reproducibly}\label{packaging-research-reproducibly}}

\label{compendia}

Containers provide a high degree of isolation that is often desirable
when attempting to capture a specific computational environment so that
others can reproduce and extend a research result. Many computationally
intensive research projects depend on specific versions of original and
third-party software packages in diverse languages, joined together to
form a pipeline through which data flows. New releases of even just a
single piece of software in this pipeline can break the entire workflow,
making it difficult to find the error and difficult for others to reuse
existing pipelines. These breakages can make the original the results
irreproducible and not extandable. The chance of a substantial
disruption like this is high in a multi-year research project where key
pieces of third-party software may have several major updates over the
duration of the project. The classical ``paper'' article is insufficient
to adequately communicate the knowledge behind such research projects
\citep[cf.][]{donoho_invitation_2010,marwick_how_2015}.

\citet{gentleman_statistical_2007} coined the term \textbf{Research
Compendium} for a dynamic document together with supporting data and
code. They use the R package system \cite{core_writing_1999} as for the
functional prototype to structuring, validation, and distribution of
research compendia. This concept has been taken up and
extended\footnote{See full literature list at \href{https://research-compendium.science/}{https://research-compendium.science/}.},
not the least by applying containerisation and other methods for
managing computing environments---see Section~\nameref{envs}. Containers
give the researcher an isolated environment to assemble these research
pipelines with specific versions of software. Research workflows in
containers are safe from contamination from other activities occuring on
the researcher's computer, for example the installation of newest
version of packages for teaching demonstrations. Given the users in this
scenario, i.e., often academics with limited formal software development
training, templates and assistance with containers around research
compendia is essential. In many fields we see that a typical unit of
research for a container is a research report or journal article, where
the container holds the compendium, or self-contained set of data (or
connections to data elsewhere) and code files needed to fully reproduce
the article \citep{marwick_packaging_2018}. The package \pkg{rrtools}
(\url{https://github.com/benmarwick/rrtools}) provides a template and
convienence functions to apply good practices for research compendia,
including a starter \texttt{Dockerfile}. Images of compendium containers
can be hosted on services such as Docker~Hub for convienient sharing
among collaborators and others. Similarly, packages such as
\pkg{containerit} and \pkg{dockerfiler} can be used to manage the
\texttt{Dockerfile} to be archived with a compendium on a data
repository (e.g.~\href{https://zenodo.org/}{Zenodo},
\href{https://dataverse.org/}{Dataverse},
\href{https://figshare.com/}{Figshare}, \href{https://osf.io/}{OSF}). A
typial compendium's \texttt{Dockerfile} will pull a rocker image fixed
to a specific version of R, and install R packages from the MRAN
repository to ensure the package versions are tied to a specific date,
rather than the most recent version. Future researchers can download the
compendium from the repository and run the included \texttt{Dockerfile}
to build a new image that recreates the computational environment used
to produce the original research results. If building the image fails,
the human-readable instructions in a \texttt{Dockerfile} are the
starting point for rebuilding the environment. Further safeguarding
practices are currently under development or not part of common practice
yet, such as preservation of images \citep{emsley_framework_2018} and
storing them alongside \texttt{Dockerfile}s
\citep[cf.][]{nust_opening_2017}, or pinning of system libraries beyond
the tagged base images, which may be seen as stable or dynamic depending
on the applied time scale
\citep[see discussion on `debian:testing` base image in][]{RJ-2017-065}.

A recommendation of the recent National Academies' report on
\emph{Reproducibility and Replicability in Science} is that journals
\emph{``consider ways to ensure computational reproducibility for
publications that make claims based on computations''}
\citep{NASEM2019}. In fields such as Political Science and Economics,
journals are increasingly adopting policies requiring authors to publish
the code and data required to reproduce computational findings reported
in published manuscripts, subject to independent verification
\citep{Jacoby2017,Vilhuber2019,Alvarez2018,Christian2018,Eubank2016,King1995}.
Problems with the computational environment, installation and
availability of software dependencies are common. R is gaining
popularity in these communities, such as creating a research compendium.
In a sample of 105 replication packages published by the \emph{American
Journal of Political Science} (AJPS) over 65\% use R. The NSF-funded
Whole Tale project uses the Rocker Project community images with the
goal of improving the reproducibility of published research artifacts
and simplifying the publication and verification process for both
authors and reviewers by reducing errors and time spent specifying the
environment.

\hypertarget{development-debugging-and-testing}{%
\subsection{Development, debugging, and
testing}\label{development-debugging-and-testing}}

\label{development}

Containers can also serve as useful playgrounds to create environments
ad-hoc or to provide very specific environments that are not needed or
not easily available in day-to-day development for the purposes of
developing R packages. These environments may have specific versions of
R, of R extension packages, and of system libraries used by R extension
packages, and all of the above in a specific combination.

First, such containers can greatly facilitate \textbf{fixing bugs and
quick evaluation}, because developers and users can readily start and
later discard a container to investigate a bug report or try out a piece
of software \citep[cf.][]{ooms_opencpu_2017}. The container does not
affect their regular system. Using the Rocker images with RStudio, these
disposable environments lack no development comfort
(cf.~Section~\nameref{compendia}). \citet{ooms_opencpu_2017} describes
how \texttt{docker\ exec} can be used to get a root shell in a container
for cusotmization during software evaluation.
\citet{eddelbuettel_debugging_2019} describes an example how a Docker
container was used to debug an issue with a package only occuring with a
particular version of Fortran, and using tools which are not readily
available on all platforms (e.g., not on macOS).

Second, the strong integration of \textbf{system libraries in core
packages} in the \href{https://www.r-spatial.org/}{R-spatial community}
makes containers essential for stable and proactive development of
common classes for geospatial data modelling and analysis. For example,
GDAL \citep{gdal_2019} is a crucial library in the geospatial domain.
GDAL is a system dependen\^{}cy allowing R packages such as
\CRANpkg{sf}, which provides the core data model for geospatial vector
data, or \CRANpkg{rgdal}, to accomodate users to be able to read and
write hundreds of different spatial raster and vector formats.
\CRANpkg{sf} and \CRANpkg{rgdal} have hundreds of indirect reverse
imports and dependencies and therefore the maintainers spend a lot of
efforts not to break these. Purpose built Docker are used to prepare for
upcoming releases of system libraries, individual bug reports, and for
the lowest supported versions of system
libraries\footnote{Cf. \href{https://github.com/r-spatial/sf/tree/master/inst/docker}{https://github.com/r-spatial/sf/tree/master/inst/docker}, \href{https://github.com/Nowosad/rspatial_proj6}{https://github.com/Nowosad/rspatial\_proj6}, and \href{https://github.com/r-spatial/sf/issues/1231}{https://github.com/r-spatial/sf/issues/1231}}.

Third, there are special purpose images for identifying problems beyond
the mere R code, such as \textbf{debugging R memory problems}. The
images significantly reduce the barrier to follow complex steps for
fixing memory allocation bugs
\citep[cf. Section~4.3 in][]{core_writing_1999}. These problems are hard
to debug and critical, both because when they occur they lead to fatal
crashing processes.
\href{https://github.com/rocker-org/r-devel-san}{\texttt{rocker/r-devel-san}}
and
\href{https://github.com/rocker-org/r-devel-san-clang}{\texttt{rocker/r-devel-ubsan-clang}}
are Docker images have a particularly configured version of R to trace
such problems with gcc and clang compilers, respectively
\citep[cf.~\CRANpkg{sanitizers} for examples,][]{eddelbuettel_sanitizers_2014}.
The image \href{https://github.com/wch/r-debug}{\texttt{wch/r-debug}} is
a purpose built Docker image with \emph{multiple} instrumented builds of
R, each with a different diagnostic utility activated.

Fourth, containers are useful for \textbf{testing} R code during
development. To submit a package to CRAN, an R package must work with
the development version of R, which must be compiled locally. That can
be a challenge for some users. The R-hub service (see
Section~\nameref{rhub}) makes it easy to ensure that no errors occur,
but to fix errors a local setup is still often warranted, e.g., using
the image \texttt{rocker/r-devel}, and to test packages with native
code, which can make the process more complex
\citep[cf.][]{eckert_building_2018}. The R-hub Docker images can also be
used to debug problems locally using various combinations of Linux
platforms, R versions, and
compilers\footnote{See \href{https://r-hub.github.io/rhub/articles/local-debugging.html}{https://r-hub.github.io/rhub/articles/local-debugging.html} and \href{https://blog.r-hub.io/2019/04/25/r-devel-linux-x86-64-debian-clang/}{https://blog.r-hub.io/2019/04/25/r-devel-linux-x86-64-debian-clang/}}.
The images go beyond the configurations, or \emph{flavours}, used by
CRAN for checking
packages\footnote{\href{https://cran.r-project.org/web/checks/check_flavors.html}{https://cran.r-project.org/web/checks/check\_flavors.html}},
e.g., with CentOS-based images, but lack a container for checking on
Windows or OS X. The images greatly support package developers to
provide support on operating systems they are not familiar. The package
\pkg{dockertest} (\url{https://github.com/traitecoevo/dockertest/}) is a
proof of concept for automatically generating \texttt{Dockerfile}s and
building images specifically to run
tests\footnote{\pkg{dockertest} is not actively maintained, but mentioned still because of its interesting approach.}.
These images are accompanied with a special launch script so the tested
source code is not stored in the image but the currently checked in
version from a local Git repository is cloned into the container at
runtime. This approach clearly seperates test environment, test code,
and current working copy of the code.

Fifth, Docker images can be used \textbf{on CI platforms} to streamline
the testing of packages. \citet{ye_docker_2019} describes how they speed
up the process of testing by running tasks on
\href{https://travis-ci.org/}{Travis~CI} within a container using
\texttt{docker\ exec}, e.g., the package check or rendering of
documentation. \citet{cardozo_faster_2018} saved time, also on
Travis~CI, by re-using the testing image as the base for an image
intended for publication on Docker~Hub.
\href{https://github.com/ColinFay/r-ci}{\texttt{r-ci}} is in turn used
with \href{https://docs.gitlab.com/ee/ci/}{GitLab~CI}, which itself is
built on top of Docker images: the user specifies a base Docker image,
and the whole tests are run inside this environment. The \texttt{r-ci}
image stack combines \texttt{rocker} versioning and a series of tools
specifically designed for testing in a fixed environment with a
customized list of preinstalled packages. Especially for long running
tests or complex system dependencies, these approaches to seperate
installation of build dependencies with code testing streamline the
development process. Not due to a concern about time, but to control the
environment used on a CI server, even this manuscript is rendered into a
PDF and deployed to a GitHub-hosted website with every change (see
\texttt{.travis.yml} and \texttt{Dockerfile} in the manuscript
repository). This gives on the one hand easy access after every update
of the R Markdown source code, and on the other hand a second controlled
environment making sure that the article renders successfully and
correctly.

\hypertarget{conclusions}{%
\section{Conclusions}\label{conclusions}}

\label{conclusions}

This article is a snapshot of the R-corner in a universe of applications
built with a many-faced piece of software, Docker. \texttt{Dockerfile}s
and Docker images are the go-to methods for collaboration between roles
in an organisation, such as development and IT operations teams, and
between parties in the communication of knowledge, such as research
workflows or education. Docker became synonymous with applying the
concept of containerisation to solve challenges of reproducible
environments, e.g., in research and in development \& production, and of
scalable deployments with the ability to move processing between
machines easily (e.g., locally, one cloud providers VM, another cloud
provider's Container-as-a-Service). Reproducible environments,
scalability and efficiency, and portability across
platforms/infrastructures are the common themes behind R packages, use
cases, and applications in this work.

The R packages and use cases presented show the growing number of users,
developers, and real-world applications in the community and the
resulting innovations. But the applications also point to the challenge
of keeping up with a continuously evolving landscape. The use cases
contributed by co-authors also have a degree of overlap, which can be
expected as a common language and understanding of good practices is
still taking shape. Also, the ease with which one can create a complex
software systems with Docker, such as an independent Docker image stack,
to serve one's specific needs leads to parallel developments. This
ease-of-DIY in combination with the difficulty to compose environments
based on \texttt{Dockerfile}s alone is a further reason for
\textbf{fragmentation}. Instructions can be outsourced into
distributable scripts and then copied into the image during build, but
that make \texttt{Dockerfile} hard to read and adds a layor of
complexity. Despite the different image stacks presented here, the
pervasiveness of Rocker can be traced back to its maintainers and the
user community valueing collaboration and shared starting points over
impulses to create individual solutions. Aside from that, fragmentation
may not be a bad sign, but instead a reflection of a growing market,
which is able to sustain multiple related efforts. With the maturing of
core building blocks, such as the Rocker suite of images, more systems
will be built successfully but will also be behind the curtains. Docker
alone, as a flexible core technology, is not a feasible level of
collaboration and abstraction. Instead, the use cases and applications
observed in this work provide a more useful division.

Nonetheless, at least on the level of R packages some
\textbf{consolidation} seems in order, e.g., to reduce the number of
packages creating \texttt{Dockerfile}s from R code or controlling the
Docker daemon with R code. It remains to be seen which approach to
control Docker, via the Docker API as \pkg{stevedore} or via system
calls as \pkg{dockyard}/\pkg{docker}/\pkg{dockr}, is more sustainable,
or if the question will be answered by the endurance of maintainers and
sufficient funding. Similarly, the capturing of environments and their
serialization in form of a \texttt{Dockerfile} currently happens at
different levels of abstraction and re-use of functionality seems
reasonable, e.g., \pkg{liftr} could generate the environment with
\pkg{containerit}, which in turn may use \pkg{dockerfiler} for low level
R objects representing a \texttt{Dockerfile} and its instructions. In
this consolidation, the Rocker~Project could play the role of
coordinating entity. Though for the moment, the sign of the times points
to more experimentation and feature growth, e.g., images for GPU-based
computing and artificial intelligence. Even with coding being more and
more accepted as a required, and achievable skill, an easier access, for
example by exposing containerisation benefits via simple user interfaces
in the users' IDE, could be an important next step. Currently
containerisation happens more in the background at the system level.

New features, which make complex workflows accessible and reproducible,
and the variety in packages connected with containerisation, even when
they have overlapping features, are a signal and a support for a growing
user base. This growth is possibly the most important goal for the
foreseeable future in the \emph{Rockerverse}, and just like the Rocker
images have matured over years of use and millions of runs, the new
ideas and prototypes will have to proof themselves. It should be noted
that the dominant position of Docker is a blessing and a curse for these
goals. It could be wise to start experimenting with non-Docker
containerisation tools now, e.g., R packages interfacing with other
container engines such as
\href{https://github.com/containers/libpod}{podman/buildah} or
\href{https://coreos.com/rkt/}{rkt}, or an R package for creating
\texttt{Singularity} files. Such efforts can help to avoid lock-in and
to design sustainable workflows based on concepts of
\emph{containerisation}, not on their implementation in Docker. If
adoption of containerisation and R continue to grow, the missing pieces
for a success predominantly lie in (a) coordination and documentation of
activities to reduce repeated work in favour of open collaboration, (b)
the sharing of lessons learned from use cases to build common knowledge
and language, and (c) a sustainable continuation and funding for all of
development, community support, and education. A concrete effort to work
towards these pieces is to sustain the structure and captured status quo
from this work in form of a \emph{CRAN Task View on containerization}.

\hypertarget{author-contributions}{%
\section{Author contributions}\label{author-contributions}}

The ordering of authors following DN as first author is alphabetical. DN
conceived the article idea,
\href{https://github.com/nuest/rockerverse-paper/issues/3}{initialised the formation of the writing team},
wrote sections not mentioned below, and revised all sections. DB wrote
the section on \pkg{outsider}. GD contributed the CARD.com use case. DE
wrote the introduction and the section about Containerization and
Rocker. RC \& EH contributed to the section on interfaces for Docker in
R. DC contributed content on Gigantum. ME contributed to the section on
processing and deployment to cloud services. CF wrote paragraphs about
\pkg{r-online}, \pkg{dockerfiler}, \pkg{r-ci} and \pkg{r-db}. SL
contributed content on RStudio's usage of Docker. BM wrote the section
on research compendia and made the project Binder-ready. HN \& JN
co-wrote the section on enterprise production environments
(\pkg{googleComputeEngineR}, Google Cloud Run, Kubernetes). KR wrote the
section about \pkg{holepunch}. NR wrote paragraphs about shared work
environments and GPUs. LS \& NT wrote the section on Bioconductor. CW
wrote the sections on Whole Tale and contributed the publication
reproducibility audit use case. NX contributed content on \pkg{liftr}.
All authors approved the final version. This articles was
collaboratively written at
\href{https://github.com/nuest/rockerverse-paper/}{https://github.com/nuest/rockerverse-paper/}.
The
\href{https://github.com/nuest/rockerverse-paper/graphs/contributors}{contributors page},
\href{https://github.com/nuest/rockerverse-paper/commits/master}{commit history},
and
\href{https://github.com/nuest/rockerverse-paper/issues/}{discussion issues}
provide a detailed view on the respective contributions.

\hypertarget{acknowledgements}{%
\section{Acknowledgements}\label{acknowledgements}}

DN is supported by the project Opening Reproducible Research
(\href{https://www.uni-muenster.de/forschungaz/project/12343}{o2r})
funded by the German Research Foundation (DFG) under project number
\href{https://gepris.dfg.de/gepris/projekt/415851837}{PE~1632/17-1}. The
funders had no role in data collection and analysis, decision to
publish, or preparation of the manuscript. CW is supported by the Whole
Tale projects (\url{https://wholetale.org}) funded by the US National
Science Foundation (NSF) under award
\href{https://www.nsf.gov/awardsearch/showAward?AWD_ID=1541450}{OAC-1541450}.

\bibliography{RJreferences}


\address{%
Daniel Nüst\\
University of Münster\\
Institute for Geoinformatics\\ Heisenbergstr. 2\\ 48149 Münster, Germany\\ \orcid{0000-0002-0024-5046}\\
}
\href{mailto:daniel.nuest@uni-muenster.de}{\nolinkurl{daniel.nuest@uni-muenster.de}}

\address{%
Dom Bennett\\
Gothenburg Global Biodiversity Centre, Sweden\\
Carl Skottsbergs gata 22B\\ 413 19 Göteborg, Sweden\\ \orcid{0000-0003-2722-1359}\\
}
\href{mailto:dominic.john.bennett@gmail.com}{\nolinkurl{dominic.john.bennett@gmail.com}}

\address{%
Robrecht Cannoodt\\
Ghent University\\
Data Mining and Modelling for Biomedicine group\\ VIB Center for Inflammation Research\\ Technologiepark 71\\ 9052 Ghent, Belgium\\ \orcid{0000-0003-3641-729X}\\
}
\href{mailto:robrecht@cannoodt.dev}{\nolinkurl{robrecht@cannoodt.dev}}

\address{%
Dav Clark\\
Gigantum, Inc.\\
1140 3rd Street NE\\ Washington, D.C. 20002, USA\\ \orcid{0000-0002-3982-4416}\\
}
\href{mailto:dav@gigantum.com}{\nolinkurl{dav@gigantum.com}}

\address{%
Gergely Daroczi\\
\\
\orcid{0000-0003-3149-8537}\\
}
\href{mailto:daroczig@rapporter.net}{\nolinkurl{daroczig@rapporter.net}}

\address{%
Dirk Eddelbuettel\\
University of Illinois at Urbana-Champaign\\
Department of Statistics\\ Illini Hall, 725 S Wright St\\ Champaign, IL 61820, USA\\ \orcid{0000-0001-6419-907X}\\
}
\href{mailto:dirkd@eddelbuettel.com}{\nolinkurl{dirkd@eddelbuettel.com}}

\address{%
Mark Edmondson\\
IIH Nordic A/S, Google Developer Expert for GCP\\
\\
}
\href{mailto:mark@markedmondson.me}{\nolinkurl{mark@markedmondson.me}}

\address{%
Colin Fay\\
ThinkR\\
5O rue Arthur Rimbaud\\ 93300 Aubervilliers, France\\ \orcid{0000-0001-7343-1846}\\
}
\href{mailto:contact@colinfay.me}{\nolinkurl{contact@colinfay.me}}

\address{%
Ellis Hughes\\
Fred Hutchinson Cancer Research Center\\
Vaccine and Infectious Disease\\ 1100 Fairview Ave. N., P.O. Box 19024\\ Seattle, WA 98109-1024, USA\\
}
\href{mailto:ehhughes@fredhutch.org}{\nolinkurl{ehhughes@fredhutch.org}}

\address{%
Sean Lopp\\
RStudio, Inc\\
250 Northern Ave\\ Boston, MA 02210, USA\\
}
\href{mailto:sean@rstudio.com}{\nolinkurl{sean@rstudio.com}}

\address{%
Ben Marwick\\
University of Washington\\
Department of Anthropology\\ Denny Hall 230, Spokane Ln\\ Seattle, WA 98105, USA\\ \orcid{0000-0001-7879-4531}\\
}
\href{mailto:bmarwick@uw.edu}{\nolinkurl{bmarwick@uw.edu}}

\address{%
Heather Nolis\\
T-Mobile\\
12920 Se 38th St.\\ Bellevue, WA, 98006, USA\\
}
\href{mailto:heather.wensler1@t-mobile.com}{\nolinkurl{heather.wensler1@t-mobile.com}}

\address{%
Jacqueline Nolis\\
Nolis, LLC\\
Seattle, WA, USA\\ \orcid{0000-0001-9354-6501}\\
}
\href{mailto:jacqueline@nolisllc.com}{\nolinkurl{jacqueline@nolisllc.com}}

\address{%
Hong Ooi\\
Microsoft\\
Level 5, 4 Freshwater Place\\ Southbank, VIC 3006, Australia\\
}
\href{mailto:hongooi@microsoft.com}{\nolinkurl{hongooi@microsoft.com}}

\address{%
Karthik Ram\\
Berkeley Institute for Data Science\\
University of California\\ Berkeley, CA 94720, USA\\ \orcid{0000-0002-0233-1757}\\
}
\href{mailto:karthik.ram@berkeley.edu}{\nolinkurl{karthik.ram@berkeley.edu}}

\address{%
Noam Ross\\
EcoHealth Alliance\\
460 W 34th St., Ste. 1701\\ New York, NY 10001, USA\\ \orcid{0000-0002-0233-1757}\\
}
\href{mailto:ross@ecohealthalliance.org}{\nolinkurl{ross@ecohealthalliance.org}}

\address{%
Lori Shepherd\\
Roswell Park Comprehensive Cancer Center\\
Elm \& Carlton Streets\\ Buffalo, NY, 14263, USA\\ \orcid{0000-0002-5910-4010}\\
}
\href{mailto:lori.shepherd@roswellpark.org}{\nolinkurl{lori.shepherd@roswellpark.org}}

\address{%
Nitesh Turaga\\
Roswell Park Comprehensive Cancer Center\\
Elm \& Carlton Streets\\ Buffalo, NY, 14263, USA\\ \orcid{0000-0002-0224-9817}\\
}
\href{mailto:nitesh.turaga@roswellpark.org}{\nolinkurl{nitesh.turaga@roswellpark.org}}

\address{%
Craig Willis\\
University of Illinois at Urbana-Champaign\\
501 E. Daniel St.\\ Champaign, IL 61820, USA\\ \orcid{0000-0002-6148-7196}\\
}
\href{mailto:willis8@illinois.edu}{\nolinkurl{willis8@illinois.edu}}

\address{%
Nan Xiao\\
Seven Bridges Genomics\\
529 Main St, Suite 6610\\ Charlestown, MA 02129, USA\\ \orcid{0000-0002-0250-5673}\\
}
\href{mailto:me@nanx.me}{\nolinkurl{me@nanx.me}}

\address{%
Charlotte Van Petegem\\
Ghent University\\
Department WE02\\ Krijgslaan 281, S9\\ 9000 Gent, Belgium\\ \orcid{0000-0003-0779-4897}\\
}
\href{mailto:charlotte.vanpetegem@ugent.be}{\nolinkurl{charlotte.vanpetegem@ugent.be}}

